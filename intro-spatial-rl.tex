\documentclass[]{article}
\usepackage{amssymb,amsmath}
\usepackage{ifxetex,ifluatex}
\ifxetex
  \usepackage{fontspec,xltxtra,xunicode}
  \defaultfontfeatures{Mapping=tex-text,Scale=MatchLowercase}
\else
  \ifluatex
    \usepackage{fontspec}
    \defaultfontfeatures{Mapping=tex-text,Scale=MatchLowercase}
  \else
    \usepackage[utf8]{inputenc}
  \fi
\fi
\usepackage{color}
\usepackage{fancyvrb}
\DefineShortVerb[commandchars=\\\{\}]{\|}
\DefineVerbatimEnvironment{Highlighting}{Verbatim}{commandchars=\\\{\}}
% Add ',fontsize=\small' for more characters per line
\newenvironment{Shaded}{}{}
\newcommand{\KeywordTok}[1]{\textcolor[rgb]{0.00,0.44,0.13}{\textbf{{#1}}}}
\newcommand{\DataTypeTok}[1]{\textcolor[rgb]{0.56,0.13,0.00}{{#1}}}
\newcommand{\DecValTok}[1]{\textcolor[rgb]{0.25,0.63,0.44}{{#1}}}
\newcommand{\BaseNTok}[1]{\textcolor[rgb]{0.25,0.63,0.44}{{#1}}}
\newcommand{\FloatTok}[1]{\textcolor[rgb]{0.25,0.63,0.44}{{#1}}}
\newcommand{\CharTok}[1]{\textcolor[rgb]{0.25,0.44,0.63}{{#1}}}
\newcommand{\StringTok}[1]{\textcolor[rgb]{0.25,0.44,0.63}{{#1}}}
\newcommand{\CommentTok}[1]{\textcolor[rgb]{0.38,0.63,0.69}{\textit{{#1}}}}
\newcommand{\OtherTok}[1]{\textcolor[rgb]{0.00,0.44,0.13}{{#1}}}
\newcommand{\AlertTok}[1]{\textcolor[rgb]{1.00,0.00,0.00}{\textbf{{#1}}}}
\newcommand{\FunctionTok}[1]{\textcolor[rgb]{0.02,0.16,0.49}{{#1}}}
\newcommand{\RegionMarkerTok}[1]{{#1}}
\newcommand{\ErrorTok}[1]{\textcolor[rgb]{1.00,0.00,0.00}{\textbf{{#1}}}}
\newcommand{\NormalTok}[1]{{#1}}
\usepackage{graphicx}
% We will generate all images so they have a width \maxwidth. This means
% that they will get their normal width if they fit onto the page, but
% are scaled down if they would overflow the margins.
\makeatletter
\def\maxwidth{\ifdim\Gin@nat@width>\linewidth\linewidth
\else\Gin@nat@width\fi}
\makeatother
\let\Oldincludegraphics\includegraphics
\renewcommand{\includegraphics}[1]{\Oldincludegraphics[width=8cm]{#1}}
\ifxetex
  \usepackage[setpagesize=false, % page size defined by xetex
              unicode=false, % unicode breaks when used with xetex
              xetex,
              colorlinks=true,
              linkcolor=blue]{hyperref}
\else
  \usepackage[unicode=true,
              colorlinks=true,
              linkcolor=blue]{hyperref}
\fi
\hypersetup{breaklinks=true, pdfborder={0 0 0}}
\setlength{\parindent}{0pt}
\setlength{\parskip}{6pt plus 2pt minus 1pt}
\setlength{\emergencystretch}{3em}  % prevent overfull lines
\setcounter{secnumdepth}{0}

\pagestyle{myheadings}
\markboth{\hfill }{GeoTALISMAN Short Course \hfill}
\author{
Cheshire, James\\
\texttt{james.cheshire@ucl.ac.uk}
\and
Lovelace, Robin\\
\texttt{r.lovelace@leeds.ac.uk}
}
\title{Introduction to Spatial Data and ggplot2}
\usepackage[margin=2cm]{geometry}

\begin{document}
\maketitle

\section{Introduction}

This tutorial is an ``Introduction to Spatial Data and ggplot2'' and
assumes no prior knowledge of spatial data analysis in R. We do
recommend users are acquainted with the R command line before beginning
the practicals though, perhaps via an `Introduction to R' type tutorial,
such as ``A (very) short introduction to R'' (Torfs and Brauer, 2012) or
the more geographically inclined ``Short introduction to R'' (Harris,
2012).

Building on such background material, the following set of exercises is
concerned with specific functions for spatial data and also the use of a
package called ggplot2 for data visualisation. An up-to-date version of
this document is maintained at
\href{https://github.com/Robinlovelace/Creating-maps-in-R/blob/master/intro-spatial-rl.pdf}{https://github.com/Robinlovelace/Creating-maps-in-R}.
Suggested improvements welcome.

\subsection{Typographic conventions}

To ensure reproducibility and allow automatic syntax highlighting, this
document has been written in RMarkdown. Be aware of the following
typographic conventions: R code (e.g. \texttt{plot(x, y)}) is written in
a \texttt{monospace} font while prose is not. Blocks of code such as,

\begin{Shaded}
\begin{Highlighting}[]
\KeywordTok{c}\NormalTok{(}\DecValTok{1}\NormalTok{:}\DecValTok{3}\NormalTok{, }\DecValTok{5}\NormalTok{)^}\DecValTok{2}
\end{Highlighting}
\end{Shaded}
\begin{verbatim}
## [1]  1  4  9 25
\end{verbatim}
are compiled in-line: the \texttt{\#\#} indicates this is output from R.
Some of the output from the code below is quite long; we only show the
output that is useful. A single hash (\texttt{\#}) is a comment for
humans to read that R will ignore. All images in this document are small
and low-quality to save space; they should display better on your
computer screen and can be saved at any resolution. The code presented
here is not the only way to do things: we encourage you to play with it
and try things out to gain a deeper understanding of R. Don't worry, you
cannot `break' anything using R and all the input data can be re-loaded
if things do go wrong.

\subsection{Prerequisites and packages}

For this tutorial you need to install R, the latest version of which can
be downloaded from
\href{http://cran.r-project.org/}{http://cran.r-project.org/}. A number
of R editors such as \href{http://www.rstudio.com/}{RStudio} can be used
to make R more user friendly, but these are not needed to complete the
tutorial.

R has a huge and growing number of spatial data packages. These can be
installed in one go with the \texttt{ctv} package and the command
\texttt{install.views("Spatial")}. We do NOT recommend running this
command for this tutorial: partly because downloading and compiling all
spatial packages takes a long time (hundreds of megabytes) and also
because we will add new packages when they are needed to see what each
does. We do recommend taking a quick browse at the range of spatial
packages on offer though:
\href{http://cran.r-project.org/web/views/Spatial.html}{http://cran.r-project.org/web/views/Spatial.html}.

The packages we will be using are \texttt{ggplot2}, \texttt{rgdal},
\texttt{rgeos}, \texttt{maptools} and \texttt{ggmap}. To test whether
ggplot2 is installed, for example, enter \texttt{library(ggpot2)}. If
you get an error message, it needs to be installed:
\texttt{install.packages("ggplot2")}.

All of the data used for the tutorial can be downloaded from here:

https://www.dropbox.com/sh/0z9a0hrn72poql5/Bx3rgWZ0kN

Save this to a new folder, then in R specify the path of that folder as
you working directory. Use the \texttt{setwd} command to do this. If
your username is ``username'' and you saved the files into a folder
called ``rmapping'' on your Desktop, for example, you would type the
following:

\begin{Shaded}
\begin{Highlighting}[]
\KeywordTok{setwd}\NormalTok{(}\StringTok{"C:/Users/username/Desktop/rmapping/R"}\NormalTok{)}
\end{Highlighting}
\end{Shaded}
If you are working in RStudio, you can create a project that will
automatically set your working directory.

\subsection{Loading spatial data}

One of the most important steps in handling spatial data with R is the
ability to read in shapefiles. There are a number of ways to do this.
The most simple is \texttt{readShapePoly()} in the \texttt{maptools}
package:

\begin{Shaded}
\begin{Highlighting}[]
\KeywordTok{library}\NormalTok{(maptools)  }\CommentTok{# load the package}
\NormalTok{sport <- }\KeywordTok{readShapePoly}\NormalTok{(}\StringTok{"london_sport.shp"}\NormalTok{)  }\CommentTok{# read in the shapefile}
\end{Highlighting}
\end{Shaded}
This method works OK, but it is no longer considered best practice since
it doesn't load in the spatial referencing information etc associated
with the shapefile. A more powerful way to read in geographical data is
to use the \texttt{rgdal} function \texttt{readOGR}, which automatically
extracts this information. This is R's interface to the ``Geospatial
Abstraction Library (GDAL)'' which is used by other open source GIS
packages such as QGIS and enables R to handle a broader range of spatial
data formats.

\begin{Shaded}
\begin{Highlighting}[]
\KeywordTok{library}\NormalTok{(rgdal)}
\NormalTok{sport <- }\KeywordTok{readOGR}\NormalTok{(}\DataTypeTok{dsn =} \StringTok{"."}\NormalTok{, }\StringTok{"london_sport"}\NormalTok{)}
\end{Highlighting}
\end{Shaded}
\begin{verbatim}
## OGR data source with driver: ESRI Shapefile 
## Source: ".", layer: "london_sport"
## with 33 features and 4 fields
## Feature type: wkbPolygon with 2 dimensions
\end{verbatim}
In the code above \texttt{dsn} is an \emph{argument} of the
\emph{function} \texttt{readOGR}. R functions have a default order of
functions, so \texttt{dsn =} does not actually need to be typed:
\texttt{readOGR(".", "london\_sport")} works the same, but it is good to
remember the meaning of each argument when beginning to use R, so we
sometimes include argument names when it is relevant. Here, \texttt{dsn}
stands for ``data source name'' which is the folder containing the
spatial data -- this was pre-specified when you set your working
directory. The next argument is a \emph{character string}, text
identifying the file required. There is no need to add a file extension.

The file contains the borough population and the percentage of the
population engaging in sporting activities and was taken from the
\href{http://data.london.gov.uk/datastore/package/active-people-survey-kpi-data-borough}{active
people survey}. The boundary data is from the
\href{http://www.ordnancesurvey.co.uk/oswebsite/opendata/}{Ordnance
Survey}.

All shapefiles have an attribute table. This is loaded with
\texttt{readOGR} and can be treated in a similar way to an R
\href{http://www.statmethods.net/input/datatypes.html}{data frame}.

R hides the geometry of spatial data unless you print the object (using
the \texttt{print()}). Let's take a look at the headings of sport, using
the following command: \texttt{names(sport)} The data contained in
spatial data are kept in a `slot' that can be accessed using the @
symbol: \texttt{sport@data}. This is useful if you do not wish to work
with the spatial components of the data at all times.

Type \texttt{summary(sport)} to get some additional information about
the data object. Spatial objects in R contain a variety of additional
information:

\begin{verbatim}
Object of class SpatialPolygonsDataFrame
Coordinates:
       min      max
x 503571.2 561941.1
y 155850.8 200932.5
Is projected: TRUE 
proj4string :
[+proj=tmerc +lat_0=49 +lon_0=-2 +k=0.9996012717 ....]
\end{verbatim}
In the above code \texttt{proj4string} represents the coordinate
reference system used in the data. In this file it has been incorrectly
specified so we can change it with the following:

\begin{Shaded}
\begin{Highlighting}[]
\KeywordTok{proj4string}\NormalTok{(sport) <- }\KeywordTok{CRS}\NormalTok{(}\StringTok{"+init=epsg:27700"}\NormalTok{)}
\end{Highlighting}
\end{Shaded}
\begin{verbatim}
## Warning: A new CRS was assigned to an object with an existing CRS:
## +proj=tmerc +lat_0=49 +lon_0=-2 +k=0.9996012717 +x_0=400000 +y_0=-100000 +ellps=airy +units=m +no_defs
## without reprojecting.
## For reprojection, use function spTransform in package rgdal
\end{verbatim}
You will see you get a warning. This is simply saying that you are
changing the coordinate reference system, not reprojecting the data.
Epsg:27700 is the code for British National Grid. If we wanted to
reproject the data into something like WGS84 for latitude and longitude
we would use the following code:

\begin{Shaded}
\begin{Highlighting}[]
\NormalTok{sport.wgs84 <- }\KeywordTok{spTransform}\NormalTok{(sport, }\KeywordTok{CRS}\NormalTok{(}\StringTok{"+init=epsg:4326"}\NormalTok{))}
\end{Highlighting}
\end{Shaded}
The different epsg codes are a bit of hassle to remember but you can
find them all at
\href{http://spatialreference.org/}{spatialreference.org}.

\section{ggplot2}

This next section of the practical introduces a slightly different
method of creating plots in R using the ggplot2 package. The package is
an implementation of the Grammar of Graphics (Wilkinson 2005) - a
general scheme for data visualization that breaks up graphs into
semantic components such as scales and layers. ggplot2 can serve as a
replacement for the base graphics in R (the functions you have been
plotting with today) and contains a number of default options that match
good visualisation practice.

The maps we produce will not be that meaningful - the focus here is on
sound visualisation with R and not sound analysis (obviously the value
of the former diminished in the absence of the latter!) Whilst the
instructions are step by step you are encouraged to deviate from them
(trying different colours for example) to get a better understanding of
what we are doing.

\texttt{ggplot2} is one of the best documented packages in R. The full
documentation for it can be found online and it is recommended you test
out the examples on your own machines and play with them:
http://docs.ggplot2.org/current/ .

Good examples of graphs can also be found on the website
\href{http://www.cookbook-r.com/Graphs/}{cookbook-r.com}.

Load the package:

\begin{Shaded}
\begin{Highlighting}[]
\KeywordTok{library}\NormalTok{(ggplot2)}
\end{Highlighting}
\end{Shaded}
It is worth noting that the basic \texttt{plot()} function requires no
data preparation but additional effort in colour selection/adding the
map key etc. \texttt{qplot()} and \texttt{ggplot()} (from the ggplot2
package) require some additional steps to format the spatial data but
select colours and add keys etc automatically. More on this later.

As a first attempt with ggplot2 we can create a scatter plot with the
attribute data in the sport object created above. Type:

\begin{Shaded}
\begin{Highlighting}[]
\NormalTok{p <- }\KeywordTok{ggplot}\NormalTok{(sport@data, }\KeywordTok{aes}\NormalTok{(Partic_Per, Pop_2001))}
\end{Highlighting}
\end{Shaded}
What you have just done is set up a ggplot object where you say where
you want the input data to come from. \texttt{sport@data} is actually a
data frame contained within the wider spatial object \texttt{sport} (the
\texttt{@} enables you to access the attribute table of the sport
shapefile). The characters inside the \texttt{aes} argument refer to the
parts of that data frame you wish to use (the variables
\texttt{Partic\_Per} and \texttt{Pop\_2001}). This has to happen within
the brackets of \texttt{aes()}, which means, roughly speaking
`aesthetics that vary'.\\If you just type p and hit enter you get the
error \texttt{No layers in plot}. This is because you have not told
ggplot what you want to do with the data. We do this by adding so-called
``geoms'', in this case \texttt{geom\_point()}.

\begin{Shaded}
\begin{Highlighting}[]
\NormalTok{p + }\KeywordTok{geom_point}\NormalTok{()}
\end{Highlighting}
\end{Shaded}
\begin{figure}[htbp]
\centering
\includegraphics{figure/A_simple_ggplot.png}
\caption{A simple ggplot}
\end{figure}

Within the brackets you can alter the nature of the points. Try
something like \texttt{p + geom\_point(colour = "red", size=2)} and
experiment.

If you want to scale the points by borough population and colour them by
sports participation this is also fairly easy by adding another
\texttt{aes()} argument.

\begin{Shaded}
\begin{Highlighting}[]
\NormalTok{p + }\KeywordTok{geom_point}\NormalTok{(}\KeywordTok{aes}\NormalTok{(}\DataTypeTok{colour =} \NormalTok{Partic_Per, }\DataTypeTok{size =} \NormalTok{Pop_2001))}
\end{Highlighting}
\end{Shaded}
The real power of ggplot2 lies in its ability to add layers to a plot.
In this case we can add text to the plot.

\begin{Shaded}
\begin{Highlighting}[]
\NormalTok{p + }\KeywordTok{geom_point}\NormalTok{(}\KeywordTok{aes}\NormalTok{(}\DataTypeTok{colour =} \NormalTok{Partic_Per, }\DataTypeTok{size =} \NormalTok{Pop_2001)) + }\KeywordTok{geom_text}\NormalTok{(}\DataTypeTok{size =} \DecValTok{2}\NormalTok{, }
    \KeywordTok{aes}\NormalTok{(}\DataTypeTok{label =} \NormalTok{name))}
\end{Highlighting}
\end{Shaded}
\begin{figure}[htbp]
\centering
\includegraphics{figure/ggplot_for_text.png}
\caption{ggplot for text}
\end{figure}

This idea of layers (or geoms) is quite different from the standard plot
functions in R, but you will find that each of the functions does a lot
of clever stuff to make plotting much easier (see the documentation for
a full list).

The following steps will create a map to show the percentage of the
population in each London Borough who regularly participate in sports
activities.

To get the shapefiles into a format that can be plotted we have to use
the \texttt{fortify()} function. Spatial objects in R have a number of
slots containing the various items of data (polygon geometry,
projection, attribute information) associated with a shapefile. Slots
can be thought of as shelves within the data object that contain the
different attributes. The ``polygons'' slot contains the geometry of the
polygons in the form of the XY coordinates used to draw the polygon
outline. The generic plot function can work out what to do with these,
ggplot2 cannot. We therefore need to extract them as a data frame. The
fortify function was written specifically for this purpose. For this to
work, either \texttt{gpclib} or \texttt{rgeos} packages must be
installed.

\begin{Shaded}
\begin{Highlighting}[]
\CommentTok{# library(gpclib); gpclibPermit() # uncomment if rgeos not installed}
\NormalTok{sport.f <- }\KeywordTok{fortify}\NormalTok{(sport, }\DataTypeTok{region =} \StringTok{"ons_label"}\NormalTok{)}
\end{Highlighting}
\end{Shaded}
\begin{verbatim}
## Loading required package: rgeos
## rgeos version: 0.2-19, (SVN revision 394)
##  GEOS runtime version: 3.3.8-CAPI-1.7.8 
##  Polygon checking: TRUE
\end{verbatim}
This step has lost the attribute information associated with the sport
object. We can add it back using the merge function (this performs a
data join). To find out how this function works look at the output of
typing \texttt{?merge}.

\begin{Shaded}
\begin{Highlighting}[]
\NormalTok{sport.f <- }\KeywordTok{merge}\NormalTok{(sport.f, sport@data, }\DataTypeTok{by.x =} \StringTok{"id"}\NormalTok{, }\DataTypeTok{by.y =} \StringTok{"ons_label"}\NormalTok{)}
\end{Highlighting}
\end{Shaded}
Take a look at the \texttt{sport.f} object to see its contents. You
should see a large data frame containing the latitude and longitude
(they are actually Easting and Northing as the data are in British
National Grid format) coordinates alongside the attribute information
associated with each London Borough. If you type \texttt{print(sport.f)}
you will just how many coordinate pairs are required! To keep the output
to a minimum, take a peak at the object just using the \texttt{head}
command:

\begin{Shaded}
\begin{Highlighting}[]
\KeywordTok{head}\NormalTok{(sport.f[, }\DecValTok{1}\NormalTok{:}\DecValTok{8}\NormalTok{])}
\end{Highlighting}
\end{Shaded}
\begin{verbatim}
##     id   long    lat order  hole piece  group           name
## 1 00AA 531027 181611     1 FALSE     1 00AA.1 City of London
## 2 00AA 531555 181659     2 FALSE     1 00AA.1 City of London
## 3 00AA 532136 182198     3 FALSE     1 00AA.1 City of London
## 4 00AA 532946 181895     4 FALSE     1 00AA.1 City of London
## 5 00AA 533411 182038     5 FALSE     1 00AA.1 City of London
## 6 00AA 533843 180794     6 FALSE     1 00AA.1 City of London
\end{verbatim}
It is now straightforward to produce a map using all the built in tools
(such as setting the breaks in the data) that ggplot2 has to offer.
\texttt{coord\_equal()} is the equivalent of asp=T in regular plots with
R:

\begin{Shaded}
\begin{Highlighting}[]
\NormalTok{Map <- }\KeywordTok{ggplot}\NormalTok{(sport.f, }\KeywordTok{aes}\NormalTok{(long, lat, }\DataTypeTok{group =} \NormalTok{group, }\DataTypeTok{fill =} \NormalTok{Partic_Per)) + }\KeywordTok{geom_polygon}\NormalTok{() + }
    \KeywordTok{coord_equal}\NormalTok{() + }\KeywordTok{labs}\NormalTok{(}\DataTypeTok{x =} \StringTok{"Easting (m)"}\NormalTok{, }\DataTypeTok{y =} \StringTok{"Northing (m)"}\NormalTok{, }\DataTypeTok{fill =} \StringTok{"% Sport Partic."}\NormalTok{) + }
    \KeywordTok{ggtitle}\NormalTok{(}\StringTok{"London Sports Participation"}\NormalTok{)}
\end{Highlighting}
\end{Shaded}
Now, just typing \texttt{Map} should result in your first ggplot-made
map of London! There is a lot going on in the code above, so think about
it line by line: what has each of the elements of code above has been
designed to do. Also note how the \texttt{aes()} components can be
combined into one set of brackets after \texttt{ggplot}, that has
relevance for all layers, rather than being broken into separate parts
as we did above. The different plot functions still know what to do with
these. The \texttt{group=group} points ggplot to the group column added
by \texttt{fortify()} and it identifies the groups of coordinates that
pertain to individual polygons (in this case London Boroughs).

The default colours are really nice but we may wish to produce the map
in black and white, which should produce a map like that shown below:

\begin{Shaded}
\begin{Highlighting}[]
\NormalTok{Map + }\KeywordTok{scale_fill_gradient}\NormalTok{(}\DataTypeTok{low =} \StringTok{"white"}\NormalTok{, }\DataTypeTok{high =} \StringTok{"black"}\NormalTok{)}
\end{Highlighting}
\end{Shaded}
\begin{figure}[htbp]
\centering
\includegraphics{figure/Greyscale_map.png}
\caption{Greyscale map}
\end{figure}

Saving plot images is also easy. You just need to use \texttt{ggsave}
after each plot, e.g. \texttt{ggsave("my\_map.pdf")} will save the map
as a pdf, with default settings. For a larger map, you could try the
following:

\begin{Shaded}
\begin{Highlighting}[]
\KeywordTok{ggsave}\NormalTok{(}\StringTok{"my_large_plot.png"}\NormalTok{, }\DataTypeTok{scale =} \DecValTok{3}\NormalTok{, }\DataTypeTok{dpi =} \DecValTok{400}\NormalTok{)}
\end{Highlighting}
\end{Shaded}
\section{Adding base maps to ggplot2 with ggmap}

ggmap is a package that uses the ggplot2 syntax as a template to create
maps with image tiles from the likes of Google and OpenStreetMap:

\begin{Shaded}
\begin{Highlighting}[]
\KeywordTok{library}\NormalTok{(ggmap)  }\CommentTok{# you may have to use install.packages to install it first}
\end{Highlighting}
\end{Shaded}
The sport object is in British National Grid but the ggmap image tiles
are in WGS84. We therefore need to uses the sport.wgs84 object created
in the reprojection operation earlier.

The first job is to calculate the bounding box (bb for short) of the
sport.wgs84 object to identify the geographic extent of the image tiles
that we need.

\begin{Shaded}
\begin{Highlighting}[]
\NormalTok{b <- }\KeywordTok{bbox}\NormalTok{(sport.wgs84)}
\NormalTok{b[}\DecValTok{1}\NormalTok{, ] <- (b[}\DecValTok{1}\NormalTok{, ] - }\KeywordTok{mean}\NormalTok{(b[}\DecValTok{1}\NormalTok{, ])) * }\FloatTok{1.05} \NormalTok{+ }\KeywordTok{mean}\NormalTok{(b[}\DecValTok{1}\NormalTok{, ])}
\NormalTok{b[}\DecValTok{2}\NormalTok{, ] <- (b[}\DecValTok{2}\NormalTok{, ] - }\KeywordTok{mean}\NormalTok{(b[}\DecValTok{2}\NormalTok{, ])) * }\FloatTok{1.05} \NormalTok{+ }\KeywordTok{mean}\NormalTok{(b[}\DecValTok{2}\NormalTok{, ])}
\CommentTok{# scale longitude and latitude (increase bb by 5% for plot) replace 1.05}
\CommentTok{# with 1.xx for an xx% increase in the plot size}
\end{Highlighting}
\end{Shaded}
This is then fed into the \texttt{get\_map} function as the location
parameter. The syntax below contains 2 functions. \texttt{ggmap} is
required to produce the plot and provides the base map data.

\begin{Shaded}
\begin{Highlighting}[]
\NormalTok{lnd.b1 <- }\KeywordTok{ggmap}\NormalTok{(}\KeywordTok{get_map}\NormalTok{(}\DataTypeTok{location =} \NormalTok{b))}
\end{Highlighting}
\end{Shaded}
\begin{verbatim}
## Warning: bounding box given to google - spatial extent only approximate.
\end{verbatim}
In much the same way as we did above we can then layer the plot with
different geoms.

First fortify the sport.wgs84 object and then merge with the required
attribute data (we already did this step to create the sport.f object).

\begin{Shaded}
\begin{Highlighting}[]
\NormalTok{sport.wgs84.f <- }\KeywordTok{fortify}\NormalTok{(sport.wgs84, }\DataTypeTok{region =} \StringTok{"ons_label"}\NormalTok{)}
\NormalTok{sport.wgs84.f <- }\KeywordTok{merge}\NormalTok{(sport.wgs84.f, sport.wgs84@data, }\DataTypeTok{by.x =} \StringTok{"id"}\NormalTok{, }\DataTypeTok{by.y =} \StringTok{"ons_label"}\NormalTok{)}
\end{Highlighting}
\end{Shaded}
We can now overlay this on our base map.

\begin{Shaded}
\begin{Highlighting}[]
\NormalTok{lnd.b1 + }\KeywordTok{geom_polygon}\NormalTok{(}\DataTypeTok{data =} \NormalTok{sport.wgs84.f, }\KeywordTok{aes}\NormalTok{(}\DataTypeTok{x =} \NormalTok{long, }\DataTypeTok{y =} \NormalTok{lat, }\DataTypeTok{group =} \NormalTok{group, }
    \DataTypeTok{fill =} \NormalTok{Partic_Per), }\DataTypeTok{alpha =} \FloatTok{0.5}\NormalTok{)}
\end{Highlighting}
\end{Shaded}
The code above contains a lot of parameters. Use the ggplot2 help pages
to find out what they are. The resulting map looks okay, but it would be
improved with a simpler base map in black and white. A design firm
called stamen provide the tiles we need and they can be brought into the
plot with the \texttt{get\_map} function:

\begin{Shaded}
\begin{Highlighting}[]
\NormalTok{lnd.b2 <- }\KeywordTok{ggmap}\NormalTok{(}\KeywordTok{get_map}\NormalTok{(}\DataTypeTok{location =} \NormalTok{b, }\DataTypeTok{source =} \StringTok{"stamen"}\NormalTok{, }\DataTypeTok{maptype =} \StringTok{"toner"}\NormalTok{, }
    \DataTypeTok{crop =} \NormalTok{T))}
\end{Highlighting}
\end{Shaded}
We can then produce the plot as before.

\begin{Shaded}
\begin{Highlighting}[]
\NormalTok{lnd.b2 + }\KeywordTok{geom_polygon}\NormalTok{(}\DataTypeTok{data =} \NormalTok{sport.wgs84.f, }\KeywordTok{aes}\NormalTok{(}\DataTypeTok{x =} \NormalTok{long, }\DataTypeTok{y =} \NormalTok{lat, }\DataTypeTok{group =} \NormalTok{group, }
    \DataTypeTok{fill =} \NormalTok{Partic_Per), }\DataTypeTok{alpha =} \FloatTok{0.5}\NormalTok{)}
\end{Highlighting}
\end{Shaded}
Finally, if we want to increase the detail of the base map, get\_map has
a zoom parameter.

\begin{Shaded}
\begin{Highlighting}[]
\NormalTok{lnd.b3 <- }\KeywordTok{ggmap}\NormalTok{(}\KeywordTok{get_map}\NormalTok{(}\DataTypeTok{location =} \NormalTok{b, }\DataTypeTok{source =} \StringTok{"stamen"}\NormalTok{, }\DataTypeTok{maptype =} \StringTok{"toner"}\NormalTok{, }
    \DataTypeTok{crop =} \NormalTok{T, }\DataTypeTok{zoom =} \DecValTok{11}\NormalTok{))}

\NormalTok{lnd.b3 + }\KeywordTok{geom_polygon}\NormalTok{(}\DataTypeTok{data =} \NormalTok{sport.wgs84.f, }\KeywordTok{aes}\NormalTok{(}\DataTypeTok{x =} \NormalTok{long, }\DataTypeTok{y =} \NormalTok{lat, }\DataTypeTok{group =} \NormalTok{group, }
    \DataTypeTok{fill =} \NormalTok{Partic_Per), }\DataTypeTok{alpha =} \FloatTok{0.5}\NormalTok{)}
\end{Highlighting}
\end{Shaded}
\begin{figure}[htbp]
\centering
\includegraphics{figure/Basemap_3.png}
\caption{Basemap 3}
\end{figure}

\section{Joining and clipping}

This section builds on the previous information on plotting and
highlights some of R's more advanced spatial functions from the
\texttt{rgeos} package. We look at joining new datasets to our data - an
attribute join - spatial joins, whereby data is added to the target
layer depending on the location of the origins and clipping.

To reaffirm our starting point, let's re-plot the only spatial dataset
in our workspace, and count the number of polygons:

\begin{Shaded}
\begin{Highlighting}[]
\KeywordTok{library}\NormalTok{(rgdal)}
\NormalTok{lnd <- }\KeywordTok{readOGR}\NormalTok{(}\DataTypeTok{dsn =} \StringTok{"."}\NormalTok{, }\StringTok{"london_sport"}\NormalTok{)}
\end{Highlighting}
\end{Shaded}
\begin{verbatim}
## OGR data source with driver: ESRI Shapefile 
## Source: ".", layer: "london_sport"
## with 33 features and 4 fields
## Feature type: wkbPolygon with 2 dimensions
\end{verbatim}
\begin{Shaded}
\begin{Highlighting}[]
\KeywordTok{plot}\NormalTok{(lnd)}
\end{Highlighting}
\end{Shaded}
\begin{figure}[htbp]
\centering
\includegraphics{figure/Plot_of_London.png}
\caption{Plot of London}
\end{figure}

\begin{Shaded}
\begin{Highlighting}[]
\KeywordTok{nrow}\NormalTok{(lnd)}
\end{Highlighting}
\end{Shaded}
\begin{verbatim}
## [1] 33
\end{verbatim}
\subsection{Downloading additional data}

Because we are using borough-level data, and boroughs are official
administrative zones, there is much data available at this level. We
will use the example of crime data to illustrate this data availability,
and join this with the current spatial dataset. As before, we can
download and import the data from within R:

\begin{Shaded}
\begin{Highlighting}[]
\CommentTok{# download.file('http://data.london.gov.uk/datafiles/crime-community-safety/mps-}
\CommentTok{# recordedcrime-borough.csv', destfile = 'mps-recordedcrime-borough.csv')}
\CommentTok{# uncomment and join the above code to download the data}

\NormalTok{crimeDat <- }\KeywordTok{read.csv}\NormalTok{(}\StringTok{"mps-recordedcrime-borough.csv"}\NormalTok{)  }\CommentTok{# flags an error}
\end{Highlighting}
\end{Shaded}
Initially, the \texttt{read.csv} command flags an error: open the raw
.csv file in a text editor such as Notepad, Notepad++ or GVIM, find the
problem and correct it. Alternatively, you can work out what the file
encoding is and use the correct argument (this is not recommended -
simpler just to edit the text file in most cases).

\begin{Shaded}
\begin{Highlighting}[]
\NormalTok{crimeDat <- }\KeywordTok{read.csv}\NormalTok{(}\StringTok{"mps-recordedcrime-borough.csv"}\NormalTok{, }\DataTypeTok{fileEncoding =} \StringTok{"UCS-2LE"}\NormalTok{)}
\KeywordTok{head}\NormalTok{(crimeDat)}
\KeywordTok{summary}\NormalTok{(crimeDat$MajorText)}
\NormalTok{crimeTheft <- crimeDat[}\KeywordTok{which}\NormalTok{(crimeDat$MajorText == }\StringTok{"Theft & Handling"}\NormalTok{), ]}
\KeywordTok{head}\NormalTok{(crimeTheft, }\DecValTok{2}\NormalTok{)  }\CommentTok{# change 2 for more rows}
\NormalTok{crimeAg <- }\KeywordTok{aggregate}\NormalTok{(CrimeCount ~ Spatial_DistrictName, }\DataTypeTok{FUN =} \StringTok{"sum"}\NormalTok{, }\DataTypeTok{data =} \NormalTok{crimeTheft)}
\KeywordTok{head}\NormalTok{(crimeAg, }\DecValTok{2}\NormalTok{)  }\CommentTok{# show the aggregated crime data}
\end{Highlighting}
\end{Shaded}
Now that we have crime data at the borough level, the challenge is to
join it by name. This is not always straightforward. Let us see which
names in the crime data match the spatial data:

\begin{Shaded}
\begin{Highlighting}[]
\NormalTok{lnd$name %in% crimeAg$Spatial_DistrictName}
\end{Highlighting}
\end{Shaded}
\begin{verbatim}
##  [1]  TRUE  TRUE  TRUE  TRUE  TRUE  TRUE  TRUE  TRUE  TRUE  TRUE  TRUE
## [12]  TRUE  TRUE  TRUE  TRUE  TRUE  TRUE  TRUE  TRUE  TRUE  TRUE  TRUE
## [23]  TRUE  TRUE  TRUE  TRUE  TRUE  TRUE  TRUE  TRUE  TRUE  TRUE FALSE
\end{verbatim}
\begin{Shaded}
\begin{Highlighting}[]
\NormalTok{lnd$name[}\KeywordTok{which}\NormalTok{(!lnd$name %in% crimeAg$Spatial_DistrictName)]}
\end{Highlighting}
\end{Shaded}
\begin{verbatim}
## [1] City of London
## 33 Levels: Barking and Dagenham Barnet Bexley Brent Bromley ... Westminster
\end{verbatim}
The first line of code above shows that all but one of the borough names
matches; the second tells us that it is City of London that is named
differently in the crime data. Look at the results (not shown here) on
your computer.

\begin{Shaded}
\begin{Highlighting}[]
\KeywordTok{levels}\NormalTok{(crimeAg$Spatial_DistrictName)}
\KeywordTok{levels}\NormalTok{(crimeAg$Spatial_DistrictName)[}\DecValTok{25}\NormalTok{] <- }\KeywordTok{as.character}\NormalTok{(lnd$name[}\KeywordTok{which}\NormalTok{(!lnd$name %in% }
    \NormalTok{crimeAg$Spatial_DistrictName)])}
\NormalTok{lnd$name %in% crimeAg$Spatial_DistrictName  }\CommentTok{# now all columns match}
\end{Highlighting}
\end{Shaded}
The above code block first identified the row with the faulty name and
then renamed the level to match the \texttt{lnd} dataset. Note that we
could not rename the variable directly, as it is stored as a factor.

We are now ready to join the datasets. It is recommended to use the
\texttt{join} function in the \texttt{plyr} package but the
\texttt{merge} function could equally be used.

\begin{Shaded}
\begin{Highlighting}[]
\KeywordTok{help}\NormalTok{(join)}
\KeywordTok{library}\NormalTok{(plyr)}
\KeywordTok{help}\NormalTok{(join)  }\CommentTok{# now help should appear}
\end{Highlighting}
\end{Shaded}
The documentation for join will be displayed if the plyr package is
loaded (if not, load or install and load it!). It requires all joining
variables to have the same name, so we will rename the variable to make
the join work:

\begin{Shaded}
\begin{Highlighting}[]
\KeywordTok{head}\NormalTok{(lnd$name)}
\KeywordTok{head}\NormalTok{(crimeAg$Spatial_DistrictName)  }\CommentTok{# the variables to join}
\NormalTok{crimeAg <- }\KeywordTok{rename}\NormalTok{(crimeAg, }\DataTypeTok{replace =} \KeywordTok{c}\NormalTok{(}\DataTypeTok{Spatial_DistrictName =} \StringTok{"name"}\NormalTok{))}
\KeywordTok{head}\NormalTok{(}\KeywordTok{join}\NormalTok{(lnd@data, crimeAg))  }\CommentTok{# test it works}
\end{Highlighting}
\end{Shaded}
\begin{verbatim}
## Joining by: name
\end{verbatim}
\begin{Shaded}
\begin{Highlighting}[]
\NormalTok{lnd@data <- }\KeywordTok{join}\NormalTok{(lnd@data, crimeAg)}
\end{Highlighting}
\end{Shaded}
\begin{verbatim}
## Joining by: name
\end{verbatim}
\subsection{Adding point data for clipping and spatial join}

In addition to joining by zone name, it is also possible to do
\href{http://help.arcgis.com/en/arcgisdesktop/10.0/help/index.html\#//00080000000q000000}{spatial
joins} in R. There are three main varieties: many-to-one - where the
values of many intersecting objects contribute to a new variable in the
main table - one-to-many, or one-to-one. Because boroughs in London are
quite large, we will conduct a many-to-one spatial join. We will be
using Tube Stations as the spatial data to join, with the aim of finding
out which and how many stations are found in each London borough.

\begin{Shaded}
\begin{Highlighting}[]
\KeywordTok{download.file}\NormalTok{(}\StringTok{"http://www.personal.leeds.ac.uk/~georl/egs/lnd-stns.zip"}\NormalTok{, }\StringTok{"lnd-stns.zip"}\NormalTok{)}
\KeywordTok{unzip}\NormalTok{(}\StringTok{"lnd-stns.zip"}\NormalTok{)}
\KeywordTok{library}\NormalTok{(rgdal)}
\NormalTok{stations <- }\KeywordTok{readOGR}\NormalTok{(}\DataTypeTok{dsn =} \StringTok{"."}\NormalTok{, }\DataTypeTok{layer =} \StringTok{"lnd-stns"}\NormalTok{, }\DataTypeTok{p4s =} \StringTok{"+init=epsg:27700"}\NormalTok{)}
\KeywordTok{proj4string}\NormalTok{(stations)  }\CommentTok{# this is the full geographical detail.}
\KeywordTok{proj4string}\NormalTok{(lnd)}
\KeywordTok{bbox}\NormalTok{(stations)}
\KeywordTok{bbox}\NormalTok{(lnd)}
\end{Highlighting}
\end{Shaded}
The above code loads the data correctly, but also shows that there are
problems with it: the Coordinate Reference System (CRS) differs from
that of our shapefile. Although OSGB 1936 (or EPSG 27700) is the
`correct' CRS for the UK, we will convert the stations dataset into
lat-long coordinates, as this is a more common CRS and enables easy base
map creation:

\begin{Shaded}
\begin{Highlighting}[]
\NormalTok{stationsWGS <- }\KeywordTok{spTransform}\NormalTok{(stations, }\DataTypeTok{CRSobj =} \KeywordTok{CRS}\NormalTok{(}\KeywordTok{proj4string}\NormalTok{(lnd)))}
\NormalTok{stations <- stationsWGS}
\KeywordTok{rm}\NormalTok{(stationsWGS)}
\KeywordTok{plot}\NormalTok{(lnd)}
\KeywordTok{points}\NormalTok{(stations[}\KeywordTok{sample}\NormalTok{(}\DecValTok{1}\NormalTok{:}\KeywordTok{nrow}\NormalTok{(stations), }\DataTypeTok{size =} \DecValTok{500}\NormalTok{), ])}
\end{Highlighting}
\end{Shaded}
\begin{figure}[htbp]
\centering
\includegraphics{figure/Sampling_and_plotting_stations.png}
\caption{Sampling and plotting stations}
\end{figure}

Now we can clearly see that the stations overlay the boroughs. The
problem is that the stations dataset is far more extensive than London
borough dataset; we need

\subsection{Clipping}

There are a number of functions that we can use to clip the points so
that only those falling within London boroughs are retained:
\texttt{?overlay ?sp::over library(rgeos) ?rgeos::gIntersects} We can
write off the first one straight away as it is depreciated by the
second. It seems that \texttt{gIntersects} can produce the same output
as \texttt{over}, based on
\href{http://gis.stackexchange.com/questions/63793/how-to-overlay-a-polygon-over-spatialpointsdataframe-and-preserving-the-spdf-dat}{discussion}
in the community, so either can be used. (See this
\href{http://stackoverflow.com/questions/15881455/how-to-clip-worldmap-with-polygon-in-r}{discussion}
for further alternatives.) In this tutorial we will use
\texttt{gIntersects}, for clipping although we could equally use
\texttt{gContains}, \texttt{gWithin} and other \texttt{g...} functions -
see rgeos help pages by typing \texttt{?gOverlaps} or other functions
for more. \texttt{gIntersects} will output information for each point,
telling us which polygon it interacts with (i.e.~the polygon it is in):

\begin{Shaded}
\begin{Highlighting}[]
\NormalTok{int <- }\KeywordTok{gIntersects}\NormalTok{(stations, lnd, }\DataTypeTok{byid =} \NormalTok{T)  }\CommentTok{# find which stations intersect }
\KeywordTok{class}\NormalTok{(int)  }\CommentTok{# it's outputed a matrix}
\KeywordTok{dim}\NormalTok{(int)  }\CommentTok{# with 33 rows (one for each zone) and 2532 cols (the points)}
\KeywordTok{summary}\NormalTok{(int[, }\KeywordTok{c}\NormalTok{(}\DecValTok{200}\NormalTok{, }\DecValTok{500}\NormalTok{)])  }\CommentTok{# not the output of this}
\KeywordTok{plot}\NormalTok{(lnd)}
\KeywordTok{points}\NormalTok{(stations[}\DecValTok{200}\NormalTok{, ], }\DataTypeTok{col =} \StringTok{"red"}\NormalTok{)  }\CommentTok{# note point id 200 is outside the zones}
\KeywordTok{points}\NormalTok{(stations[}\DecValTok{500}\NormalTok{, ], }\DataTypeTok{col =} \StringTok{"green"}\NormalTok{)  }\CommentTok{# note point 500 is inside}
\KeywordTok{which}\NormalTok{(int[, }\DecValTok{500}\NormalTok{] == T)  }\CommentTok{# this tells us that point 500 intersects with zone 32}
\KeywordTok{points}\NormalTok{(}\KeywordTok{coordinates}\NormalTok{(lnd[}\DecValTok{32}\NormalTok{, ]), }\DataTypeTok{col =} \StringTok{"black"}\NormalTok{)  }\CommentTok{# test the previous statement}
\end{Highlighting}
\end{Shaded}
\begin{figure}[htbp]
\centering
\includegraphics{figure/Identifying_and_plotting_individual_stations.png}
\caption{Identifying and plotting individual stations}
\end{figure}

In the above code, only the first line actually `does' anything in our
workspace, by creating the object \texttt{int}. The proceeding lines are
dedicated to exploring this object and what it means. Note that it is a
matrix with columns corresponding to the points and rows corresponding
to boroughs. The borough in which a particular point can be extracted
from \texttt{int} as we shall see below. For the purposes of clipping,
we are only interested in whether the point intersects with \emph{any}
of the boroughs. This is where the function \texttt{apply}, which is
unique to R, comes into play:

\begin{Shaded}
\begin{Highlighting}[]
\NormalTok{clipped <- }\KeywordTok{apply}\NormalTok{(int == F, }\DataTypeTok{MARGIN =} \DecValTok{2}\NormalTok{, all)}
\KeywordTok{plot}\NormalTok{(stations[}\KeywordTok{which}\NormalTok{(clipped), ])  }\CommentTok{# shows all stations we DO NOT want}
\NormalTok{stations.cl <- stations[}\KeywordTok{which}\NormalTok{(!clipped), ]  }\CommentTok{# use ! to select the invers}
\KeywordTok{points}\NormalTok{(stations.cl, }\DataTypeTok{col =} \StringTok{"green"}\NormalTok{)  }\CommentTok{# check that it's worked}
\end{Highlighting}
\end{Shaded}
\begin{figure}[htbp]
\centering
\includegraphics{figure/Clipped_points__within_London_boroughs_.png}
\caption{Clipped points (within London boroughs)}
\end{figure}

\begin{Shaded}
\begin{Highlighting}[]
\NormalTok{stations <- stations.cl}
\KeywordTok{rm}\NormalTok{(stations.cl)  }\CommentTok{# tidy up: we're only interested in clipped ones}
\end{Highlighting}
\end{Shaded}
The first line instructs R to look at each column (\texttt{MARGIN = 2},
we would use \texttt{MARGIN = 1} for row-by-row analysis) and report
back whether \texttt{all} of the values are false. This creates the
inverse selection that we want, hence the use of \texttt{!} to invert
it. We test that the function works on a new object (often a good idea,
to avoid overwriting useful data) with plots and, once content that the
clip has worked, save the sample of points to our main \texttt{stations}
object and remove the now duplicated \texttt{stations.cl} object.

\subsection{Aggregating the data to complete the spatial join}

Now that we know how \texttt{gIntersects} works in general terms and for
clipping, let's use it to allocate a borough to each of our station
points, which we will then aggregate up. Data from these points
(e.g.~counts, averages in each area etc.) can then be transferred to the
main polygons table: the essence of a spatial join. Again,
\texttt{apply} is our friend in this instance, meaning we can avoid
\texttt{for} loops:

\begin{Shaded}
\begin{Highlighting}[]
\NormalTok{int <- }\KeywordTok{gIntersects}\NormalTok{(stations, lnd, }\DataTypeTok{byid =} \NormalTok{T)  }\CommentTok{# re-run the intersection query }
\KeywordTok{head}\NormalTok{(}\KeywordTok{apply}\NormalTok{(int, }\DataTypeTok{MARGIN =} \DecValTok{2}\NormalTok{, }\DataTypeTok{FUN =} \NormalTok{which))}
\NormalTok{b.indexes <- }\KeywordTok{which}\NormalTok{(int, }\DataTypeTok{arr.ind =} \NormalTok{T)}
\KeywordTok{summary}\NormalTok{(b.indexes)}
\NormalTok{b.names <- lnd$name[b.indexes[, }\DecValTok{1}\NormalTok{]]}
\NormalTok{b.count <- }\KeywordTok{aggregate}\NormalTok{(b.indexes ~ b.names, }\DataTypeTok{FUN =} \NormalTok{length)}
\KeywordTok{head}\NormalTok{(b.count)}
\end{Highlighting}
\end{Shaded}
The above code first extracts the index of the row (borough) for which
the corresponding column is true and then converts this into names. The
final object created, \texttt{b.count} contains the number of station
points in each zone. According to this, Barking and Dagenham should
contain 12 station points. It is important to check the output makes
sense at every stage with R, so let's check to see this is indeed the
case with a quick plot:

\begin{Shaded}
\begin{Highlighting}[]
\KeywordTok{plot}\NormalTok{(lnd[}\KeywordTok{which}\NormalTok{(}\KeywordTok{grepl}\NormalTok{(}\StringTok{"Barking"}\NormalTok{, lnd$name)), ])}
\KeywordTok{points}\NormalTok{(stations)}
\end{Highlighting}
\end{Shaded}
\begin{figure}[htbp]
\centering
\includegraphics{figure/Train/tube_stations_in_Barking_and_Dagenham.png}
\caption{Train/tube stations in Barking and Dagenham}
\end{figure}

Now the fun part: count the points in the polygon and report back how
many there are!

The final stage is to transfer the data on station counts back into the
polygon data frame. We have used \texttt{merge} to join two datasets
before. In R there is often more than one way to achieve the same
result. It's good to experiment with different functions, so we will use
\texttt{join} from the \texttt{plyr} package. \texttt{join} requires
identical joining names in both data frames, so first we will rename
them (type \texttt{?rename} for more details).

\begin{Shaded}
\begin{Highlighting}[]
\NormalTok{b.count <- }\KeywordTok{rename}\NormalTok{(b.count, }\DataTypeTok{replace =} \KeywordTok{c}\NormalTok{(}\DataTypeTok{b.names =} \StringTok{"name"}\NormalTok{))}
\NormalTok{b.count.tmp <- }\KeywordTok{join}\NormalTok{(lnd@data, b.count)}
\end{Highlighting}
\end{Shaded}
\begin{verbatim}
## Joining by: name
\end{verbatim}
\begin{Shaded}
\begin{Highlighting}[]
\KeywordTok{head}\NormalTok{(b.count.tmp, }\DecValTok{2}\NormalTok{)}
\end{Highlighting}
\end{Shaded}
\begin{verbatim}
##   ons_label                 name Partic_Per Pop_2001 CrimeCount row col
## 1      00AF              Bromley       21.7   295535      15172  54  54
## 2      00BD Richmond upon Thames       26.6   172330       9715  22  22
\end{verbatim}
\begin{Shaded}
\begin{Highlighting}[]
\NormalTok{lnd$station.count <- b.count.tmp[, }\DecValTok{7}\NormalTok{]}
\end{Highlighting}
\end{Shaded}
We have now seen how to join and clip data. Next, for a stronger
grounding in how ggplot works, we will look at plotting non-spatial
data.

\section{Using ggplot2 for Descriptive Statistics}

For this we will use a new dataset:

\begin{Shaded}
\begin{Highlighting}[]
\NormalTok{input <- }\KeywordTok{read.csv}\NormalTok{(}\StringTok{"ambulance_assault.csv"}\NormalTok{)}
\end{Highlighting}
\end{Shaded}
This contains the number of ambulance callouts to assault incidents
(downloadable from the London DataStore) between 2009 and 2011.

Take a look at the contents of the file:

\begin{Shaded}
\begin{Highlighting}[]
\KeywordTok{head}\NormalTok{(input)}
\end{Highlighting}
\end{Shaded}
\begin{verbatim}
##   Bor_Code     WardName WardCode assault_09_11
## 1     00AA   Aldersgate   00AAFA            10
## 2     00AA      Aldgate   00AAFB             0
## 3     00AA    Bassishaw   00AAFC             0
## 4     00AA Billingsgate   00AAFD             0
## 5     00AA  Bishopsgate   00AAFE           188
## 6     00AA Bread Street   00AAFF             0
\end{verbatim}
We can now plot a histogram to show the distribution of values.

\begin{Shaded}
\begin{Highlighting}[]
\NormalTok{p.ass <- }\KeywordTok{ggplot}\NormalTok{(input, }\KeywordTok{aes}\NormalTok{(}\DataTypeTok{x =} \NormalTok{assault_09_11))}
\end{Highlighting}
\end{Shaded}
Remember the \texttt{ggplot(input, aes(x=assault\_09\_11))} section
means create a generic plot object (called p.ass) from the input object
using the \texttt{assault\_09\_11} column as the data for the x axis. To
create the histogram you need to tell R that this is what you want to go
with

\begin{Shaded}
\begin{Highlighting}[]
\NormalTok{p.ass + }\KeywordTok{geom_histogram}\NormalTok{()}
\end{Highlighting}
\end{Shaded}
The resulting message
(\texttt{stat\_bin: binwidth defaulted to range/30...}) relates to the
bins - the breaks between histogram blocks. If you want the bins (and
therefore the bars) to be thinner (i.e.~representing fewer values) you
need to make the bins smaller by adjusting the binwidth. Try:

\begin{Shaded}
\begin{Highlighting}[]
\NormalTok{p.ass + }\KeywordTok{geom_histogram}\NormalTok{(}\DataTypeTok{binwidth =} \DecValTok{10}\NormalTok{) + }\KeywordTok{geom_density}\NormalTok{(}\DataTypeTok{fill =} \OtherTok{NA}\NormalTok{, }\DataTypeTok{colour =} \StringTok{"black"}\NormalTok{)}
\end{Highlighting}
\end{Shaded}
It is also possible to overlay a density distribution over the top of
the histogram. For this we need to produce a second plot object with the
density distribution as the y variable.

\begin{Shaded}
\begin{Highlighting}[]
\NormalTok{p2.ass <- }\KeywordTok{ggplot}\NormalTok{(input, }\KeywordTok{aes}\NormalTok{(}\DataTypeTok{x =} \NormalTok{assault_09_11, }\DataTypeTok{y =} \NormalTok{..density..))}

\NormalTok{p2.ass + }\KeywordTok{geom_histogram}\NormalTok{() + }\KeywordTok{geom_density}\NormalTok{(}\DataTypeTok{fill =} \OtherTok{NA}\NormalTok{, }\DataTypeTok{colour =} \StringTok{"red"}\NormalTok{)}
\end{Highlighting}
\end{Shaded}
\begin{verbatim}
## stat_bin: binwidth defaulted to range/30. Use 'binwidth = x' to adjust this.
\end{verbatim}
\begin{figure}[htbp]
\centering
\includegraphics{figure/Histogram_with_density_overlay.png}
\caption{Histogram with density overlay}
\end{figure}

What kind of distribution is this plot showing? You can see that there
are a few wards with very high assault incidences (over 750). To find
out which ones these are we can select them.

\begin{Shaded}
\begin{Highlighting}[]
\NormalTok{input[}\KeywordTok{which}\NormalTok{(input$assault_09_11 > }\DecValTok{750}\NormalTok{), ]}
\end{Highlighting}
\end{Shaded}
\begin{verbatim}
##     Bor_Code   WardName WardCode assault_09_11
## 153     00AH  Fairfield   00AHGM           765
## 644     00BK St James's   00BKGQ          1582
## 649     00BK   West End   00BKGW          1305
\end{verbatim}
It is perhaps unsurprising that St James's and the West End have the
highest counts. The plot has provided a good impression of the overall
distribution, but what are the characteristics of each distribution
within the Boroughs? Another type of plot that shows the core
characteristics of the distribution is a box and whisker plot. These too
can be easily produced in R (you can't do them in Excel!). We can create
a third plot object (note that the assault field is now y and not x):

\begin{Shaded}
\begin{Highlighting}[]
\NormalTok{p3.ass <- }\KeywordTok{ggplot}\NormalTok{(input, }\KeywordTok{aes}\NormalTok{(}\DataTypeTok{x =} \NormalTok{Bor_Code, }\DataTypeTok{y =} \NormalTok{assault_09_11))}
\end{Highlighting}
\end{Shaded}
and convert it to a boxplot.

\begin{Shaded}
\begin{Highlighting}[]
\NormalTok{p3.ass + }\KeywordTok{geom_boxplot}\NormalTok{()}
\end{Highlighting}
\end{Shaded}
Perhaps this would look a little better flipped round.

\begin{Shaded}
\begin{Highlighting}[]
\NormalTok{p3.ass + }\KeywordTok{geom_boxplot}\NormalTok{() + }\KeywordTok{coord_flip}\NormalTok{()}
\end{Highlighting}
\end{Shaded}
\begin{figure}[htbp]
\centering
\includegraphics{figure/Bar_and_whisker_plot.png}
\caption{Bar and whisker plot}
\end{figure}

Now each of the borough codes can be easily seen. No surprise that the
Borough of Westminster (00BK) has the two largest outliers. In one line
of code you have produced an incredibly complex plot rich in
information. This demonstrates why R is such a useful program for these
kinds of statistics.

If you want an insight into some of the visualisations you can develop
with this type of data we can do faceting based on the example of the
histogram plot above.

\begin{Shaded}
\begin{Highlighting}[]
\NormalTok{p.ass + }\KeywordTok{geom_histogram}\NormalTok{() + }\KeywordTok{facet_wrap}\NormalTok{(~Bor_Code)}
\end{Highlighting}
\end{Shaded}
\begin{figure}[htbp]
\centering
\includegraphics{figure/Faceted_histogram.png}
\caption{Faceted histogram}
\end{figure}

We need to do a little bit of tweaking to make this plot publishable but
we want to demonstrate that it is really easy to produce 30+ plots on a
single page! Faceting is an extremely powerful way of visualizing
multidimensional datasets and is especially good for showing change over
time.

\section{Advanced Task: Faceting for Maps}

\begin{Shaded}
\begin{Highlighting}[]
\KeywordTok{library}\NormalTok{(reshape2)  }\CommentTok{# this may not be installed. }
\CommentTok{# If not install it, or skip the next two steps…}
\end{Highlighting}
\end{Shaded}
Load the data - this shows historic population values between 1801 and
2001 for London, again from the London data store.

\begin{Shaded}
\begin{Highlighting}[]
\NormalTok{london.data <- }\KeywordTok{read.csv}\NormalTok{(}\StringTok{"census-historic-population-borough.csv"}\NormalTok{)}
\end{Highlighting}
\end{Shaded}
``Melt'' the data so that the columns become rows.

\begin{Shaded}
\begin{Highlighting}[]
\NormalTok{london.data.melt <- }\KeywordTok{melt}\NormalTok{(london.data, }\DataTypeTok{id =} \KeywordTok{c}\NormalTok{(}\StringTok{"Area.Code"}\NormalTok{, }\StringTok{"Area.Name"}\NormalTok{))}
\end{Highlighting}
\end{Shaded}
Only do this step if reshape and melt failed

\begin{Shaded}
\begin{Highlighting}[]
\NormalTok{london.data.melt <- }\KeywordTok{read.csv}\NormalTok{(}\StringTok{"london_data_melt.csv"}\NormalTok{)}
\end{Highlighting}
\end{Shaded}
Merge the population data with the London borough geometry contained
within our sport.f object.

\begin{Shaded}
\begin{Highlighting}[]
\NormalTok{plot.data <- }\KeywordTok{merge}\NormalTok{(sport.f, london.data.melt, }\DataTypeTok{by.x =} \StringTok{"id"}\NormalTok{, }\DataTypeTok{by.y =} \StringTok{"Area.Code"}\NormalTok{)}
\end{Highlighting}
\end{Shaded}
Reorder this data (ordering is important for plots).

\begin{Shaded}
\begin{Highlighting}[]
\NormalTok{plot.data <- plot.data[}\KeywordTok{order}\NormalTok{(plot.data$order), ]}
\end{Highlighting}
\end{Shaded}
We can now use faceting to produce one map per year (this may take a
little while to appear).

\begin{Shaded}
\begin{Highlighting}[]
\KeywordTok{ggplot}\NormalTok{(}\DataTypeTok{data =} \NormalTok{plot.data, }\KeywordTok{aes}\NormalTok{(}\DataTypeTok{x =} \NormalTok{long, }\DataTypeTok{y =} \NormalTok{lat, }\DataTypeTok{fill =} \NormalTok{value, }\DataTypeTok{group =} \NormalTok{group)) + }
    \KeywordTok{geom_polygon}\NormalTok{() + }\KeywordTok{geom_path}\NormalTok{(}\DataTypeTok{colour =} \StringTok{"grey"}\NormalTok{, }\DataTypeTok{lwd =} \FloatTok{0.1}\NormalTok{) + }\KeywordTok{coord_equal}\NormalTok{() + }
    \KeywordTok{facet_wrap}\NormalTok{(~variable)}
\end{Highlighting}
\end{Shaded}
\begin{figure}[htbp]
\centering
\includegraphics{figure/Faceted_map.png}
\caption{Faceted map}
\end{figure}

Again there is a lot going on here so explore the documentation to make
sure you understand it. Try out different colour values as well.

Add a title and replace the axes names with ``easting'' and ``northing''
and save your map as a pdf.

\section{Taking spatial data analysis in R further}

The skills you have learned in this tutorial are applicable to a very
wide range of datasets, spatial or not. Often experimentation is the
most rewarding learning method, rather than just searching for the
`best' way of doing something (Kabakoff, 2011). We recommend you play
around with your own data.

The supportive online communities surrounding large open source programs
such as R are one of their greatest assets, so we recommend you become
an active
``\href{http://blog.cleverelephant.ca/2013/10/being-open-source-citizen.html}{open
source citizen}'' rather than a passive consumer (Ramsey \& Dubovsky,
2013).

This does not necessarily mean writing R source code - it can simply
mean helping others use R. We therefore conclude the tutorial with a
list of resources that will help you further sharpen you R skills, find
help and contribute to the growing online R community:

\begin{itemize}
\item
  R's homepage hosts a wealth of
  \href{http://cran.r-project.org/manuals.html}{official} and
  \href{http://cran.r-project.org/other-docs.html}{contributed} guides.
\item
  Stack Exchange and GIS Stack Exchange groups - try searching for
  ``{[}R{]}''. If your issue has not been not been addressed yet, you
  could post a polite question.
\item
  R's \href{http://www.r-project.org/mail.html}{mailing lists} - the
  R-sig-geo list may be of particular interest here.
\end{itemize}
Books: despite the strength of R's online community, nothing beats a
physical book for concentrated learning. We would particularly recommend
the following:

\begin{itemize}
\item
  ggplot2: elegant graphics for data analysis (Wickham 2009)
\item
  Bivand et al. (2013) Provide a dense and detailed overview of spatial
  data analysis in an updated version of the book by the developers of
  many of R's spatial functions.
\item
  Kabacoff (2011) is a more general R book; it has many fun worked
  examples.
\end{itemize}
\newpage \section{References}

Bivand, R. S., Pebesma, E. J., \& Rubio, V. G. (2008). Applied spatial
data: analysis with R. Springer.

Harris, R. (2012). A Short Introduction to R.
\href{http://www.social-statistics.org/}{social-statistics.org}.

Kabacoff, R. (2011). R in Action. Manning Publications Co.

Ramsey, P., \& Dubovsky, D. (2013). Geospatial Software's Open Future.
GeoInformatics, 16(4).

Torfs and Brauer (2012). A (very) short Introduction to R. The
Comprehensive R Archive Network.

Wickham, H. (2009). ggplot2: elegant graphics for data analysis.
Springer.

Wilkinson, L. (2005). The grammar of graphics. Springer.

\begin{Shaded}
\begin{Highlighting}[]
\KeywordTok{source}\NormalTok{(}\StringTok{"latex/rmd2pdf.R"}\NormalTok{)  }\CommentTok{# convert .Rmd to .tex file}
\end{Highlighting}
\end{Shaded}

\end{document}
