\documentclass[]{article}
\usepackage{amssymb,amsmath}
\usepackage{ifxetex,ifluatex}
\ifxetex
  \usepackage{fontspec,xltxtra,xunicode}
  \defaultfontfeatures{Mapping=tex-text,Scale=MatchLowercase}
\else
  \ifluatex
    \usepackage{fontspec}
    \defaultfontfeatures{Mapping=tex-text,Scale=MatchLowercase}
  \else
    \usepackage[utf8]{inputenc}
  \fi
\fi
\usepackage{color}
\usepackage{fancyvrb}
\DefineShortVerb[commandchars=\\\{\}]{\|}
\DefineVerbatimEnvironment{Highlighting}{Verbatim}{commandchars=\\\{\}}
% Add ',fontsize=\small' for more characters per line
\newenvironment{Shaded}{}{}
\newcommand{\KeywordTok}[1]{\textcolor[rgb]{0.00,0.44,0.13}{\textbf{{#1}}}}
\newcommand{\DataTypeTok}[1]{\textcolor[rgb]{0.56,0.13,0.00}{{#1}}}
\newcommand{\DecValTok}[1]{\textcolor[rgb]{0.25,0.63,0.44}{{#1}}}
\newcommand{\BaseNTok}[1]{\textcolor[rgb]{0.25,0.63,0.44}{{#1}}}
\newcommand{\FloatTok}[1]{\textcolor[rgb]{0.25,0.63,0.44}{{#1}}}
\newcommand{\CharTok}[1]{\textcolor[rgb]{0.25,0.44,0.63}{{#1}}}
\newcommand{\StringTok}[1]{\textcolor[rgb]{0.25,0.44,0.63}{{#1}}}
\newcommand{\CommentTok}[1]{\textcolor[rgb]{0.38,0.63,0.69}{\textit{{#1}}}}
\newcommand{\OtherTok}[1]{\textcolor[rgb]{0.00,0.44,0.13}{{#1}}}
\newcommand{\AlertTok}[1]{\textcolor[rgb]{1.00,0.00,0.00}{\textbf{{#1}}}}
\newcommand{\FunctionTok}[1]{\textcolor[rgb]{0.02,0.16,0.49}{{#1}}}
\newcommand{\RegionMarkerTok}[1]{{#1}}
\newcommand{\ErrorTok}[1]{\textcolor[rgb]{1.00,0.00,0.00}{\textbf{{#1}}}}
\newcommand{\NormalTok}[1]{{#1}}
\usepackage{graphicx}
% We will generate all images so they have a width \maxwidth. This means
% that they will get their normal width if they fit onto the page, but
% are scaled down if they would overflow the margins.
\makeatletter
\def\maxwidth{\ifdim\Gin@nat@width>\linewidth\linewidth
\else\Gin@nat@width\fi}
\makeatother
\let\Oldincludegraphics\includegraphics
\renewcommand{\includegraphics}[1]{\Oldincludegraphics[width=8cm]{#1}}
\ifxetex
  \usepackage[setpagesize=false, % page size defined by xetex
              unicode=false, % unicode breaks when used with xetex
              xetex,
              colorlinks=true,
              linkcolor=blue]{hyperref}
\else
  \usepackage[unicode=true,
urlcolor=blue,
              colorlinks=true,
              linkcolor=blue]{hyperref}
\fi
\hypersetup{breaklinks=true, pdfborder={0 0 0}}
\setlength{\parindent}{0pt}
\setlength{\parskip}{6pt plus 2pt minus 1pt}
\setlength{\emergencystretch}{3em}  % prevent overfull lines
\setcounter{secnumdepth}{0}

\pagestyle{myheadings}
\author{Lovelace, Robin\\
\texttt{r.lovelace@leeds.ac.uk}
\and
Cheshire, James\\
\texttt{james.cheshire@ucl.ac.uk}
}
\title{Introduction to visualising spatial data in R}
\markboth{\hfill }{GeoTALISMAN Short Course \hfill}
\usepackage[margin=2cm]{geometry}

\begin{document}
\maketitle

\tableofcontents
\section{Part I: Introduction}

This tutorial is an introduction to spatial data in R and map making
with R's `base' graphics and the popular graphics package
\texttt{ggplot2}. It assumes no prior knowledge of spatial data analysis
but prior understanding of the R command line would be beneficial. For
people new to R, we recommend working through an `Introduction to R'
type tutorial, such as ``A (very) short introduction to R''
(\href{http://cran.r-project.org/doc/contrib/Torfs+Brauer-Short-R-Intro.pdf}{Torfs
and Brauer, 2012}) or the more geographically inclined ``Short
introduction to R''
(\href{http://www.social-statistics.org/wp-content/uploads/2012/12/intro\_to\_R1.pdf}{Harris,
2012}).

Building on such background material, the following set of exercises is
concerned with specific functions for spatial data and visualisation. It
is divided into five parts:

\begin{itemize}
\item
  Introduction, which provides a guide to R's syntax and preparing for
  the tutorial
\item
  Spatial data in R, which describes basic spatial functions in R
\item
  Manipulating spatial data, which includes changing projection,
  clipping and spatial joins
\item
  Map making with \texttt{ggplot2}, a recent graphics package for
  producing beautiful maps quickly
\item
  Taking spatial analysis in R further, a compilation of resources for
  furthering your skills
\end{itemize}
An up-to-date version of this tutorial is maintained at
\href{https://github.com/Robinlovelace/Creating-maps-in-R/blob/master/intro-spatial-rl.pdf}{https://github.com/Robinlovelace/Creating-maps-in-R}
and the entire tutorial, including the input data can be downloaded as a
\href{https://github.com/Robinlovelace/Creating-maps-in-R/archive/master.zip}{zip
file}, as described below. The entire tutorial was written in RMarkdown,
which allows R code to run as the document compiles. Thus all the
examples are entirely reproducible.

Suggested improvements welcome - please
\href{https://help.github.com/articles/fork-a-repo}{fork}, improve and
push this document to its original home to ensure its longevity. The
tutorial was developed for a series of Short Courses put on by the
National Centre for Research Methods (NCRM), via the TALISMAN node (see
\href{http://www.geotalisman.org/}{geotalisman.org}).

\subsection{Typographic conventions and getting help}

To ensure reproducibility and allow automatic syntax highlighting, this
document has been written in RMarkdown. We try to follow best practice
in terms of style, roughly following Google's style guide and an
in-depth guide written by
\href{http://cran.r-project.org/web/packages/rockchalk/vignettes/Rstyle.pdf}{Johnson
(2013)}. It is a good idea to get into the habit of consistent and clear
writing in any language, and R is no exception. Adding comments to your
code is also good practice, so you remember at a later date what you've
done, aiding the learning process. There are two main ways of commenting
code using the \texttt{\#} symbol: above a line of code or directly
following it, as illustrated below.

\begin{Shaded}
\begin{Highlighting}[]
\CommentTok{# Generate data}
\NormalTok{x <- }\DecValTok{1}\NormalTok{:}\DecValTok{400}
\NormalTok{y <- }\KeywordTok{sin}\NormalTok{(x/}\DecValTok{10}\NormalTok{) * }\KeywordTok{exp}\NormalTok{(x * -}\FloatTok{0.01}\NormalTok{)}

\KeywordTok{plot}\NormalTok{(x, y)  }\CommentTok{# plot the result}
\end{Highlighting}
\end{Shaded}
\begin{figure}[htbp]
\centering
\includegraphics{figure/Basic_plot_of_x_and_y.png}
\caption{Basic plot of x and y}
\end{figure}

In the above code we first created a new \emph{object} that we have
called \texttt{x}. Any name could have been used, like \texttt{xBumkin},
but \texttt{x} works just fine here, although it is good practice to
give your objects meaningful names. Note the use of the
\texttt{\textless{}-} ``arrow'' symbol, which tells R to create a new
object. We will be using this symbol a lot in the tutorial (tip: typing
\texttt{Alt -} on the keyboard will create it in RStudio.). Each time it
is used, a new object is created (or an old one is overwritten) with a
name of your choosing.

To distinguish between prose and code, please be aware of the following
typographic conventions: R code (e.g. \texttt{plot(x, y)}) is written in
a \texttt{monospace} font while prose is not. Blocks of code such as:

\begin{Shaded}
\begin{Highlighting}[]
\KeywordTok{c}\NormalTok{(}\DecValTok{1}\NormalTok{:}\DecValTok{3}\NormalTok{, }\DecValTok{5}\NormalTok{)^}\DecValTok{2}
\end{Highlighting}
\end{Shaded}
\begin{verbatim}
## [1]  1  4  9 25
\end{verbatim}
are compiled in-line: the \texttt{\#\#} indicates this is output from R.
Some of the output from the code below is quite long so we only show the
output that is useful - it should also be clear when we have decided to
omit an image from this document to save space. All images in this
document are small and low-quality to save space; they should display
better on your computer screen and can be saved at any resolution. The
code presented here is not the only way to do things: we encourage you
to play with it and try things out to gain a deeper understanding of R.
Don't worry, you cannot `break' anything using R and all the input data
can be re-loaded if things do go wrong.

If you require help on any function, use the \texttt{help} function,
e.g. \texttt{help(plot)}. Because R users love being concise, this can
also be written as \texttt{?plot}. Feel free to use it at any point
you'd like more detail on a specific function (although R's help files
are famously cryptic for the un-initiated). Help on more general terms
can be found using the \texttt{??} symbol. To test this, try typing
\texttt{??regression}. For the most part, \emph{learning by doing} is a
good motto, so let's crack on and download some packages and then some
data.

\subsection{Prerequisites and packages}

For this tutorial you need to install R, if you haven't already done so,
the latest version of which can be downloaded from
\href{http://cran.r-project.org/}{http://cran.r-project.org/}. A number
of R editors such as \href{http://www.rstudio.com/}{RStudio} can be used
to make R more user friendly, but these are not needed to complete the
tutorial.

R has a huge and growing number of spatial data packages. We recommend
taking a quick browse on R's main website:
\href{http://cran.r-project.org/web/views/Spatial.html}{http://cran.r-project.org/web/views/Spatial.html}.

The packages we will be using are \texttt{ggplot2}, \texttt{rgdal},
\texttt{rgeos}, \texttt{maptools} and \texttt{ggmap}. To test whether a
package is installed, ggplot2 for example, enter
\texttt{library(ggplot2)}. If you get an error message, it needs to be
installed: \texttt{install.packages("ggplot2")}. These will be
downloaded from CRAN (the Comprehensive R Archive Network); if you are
prompted to select a `mirror', select one that is close to your home. If
there is no output from R, this is good news: it means that the library
has already been installed on your computer. Install these packages now.

\section{Part II: Spatial data in R}

\subsection{Starting the tutorial}

Now that we have taken a look at R's syntax and installed the necessary
packages, we can start looking at some real spatial data. This second
part introduces some spatial datasets that we will download from the
internet. Plotting these datasets and interrogating the attribute data
form the foundation of spatial data analysis in R, so we will focus on
these elements in the next two parts of the tutorial, before focussing
on creating attractive maps in Part IV.

\subsection{Downloading the data}

Download the data for this tutorial now from :
\href{https://github.com/Robinlovelace/Creating-maps-in-R}{https://github.com/Robinlovelace/Creating-maps-in-R}.
Click on the ``Download ZIP'' button on the right hand side and once it
is downloaded unzip this to a new folder on your PC. Use the
\texttt{setwd} command to set the working directory to the folder where
the data is saved. If your username is ``username'' and you saved the
files into a folder called ``Creating-maps-in-R-master'' on your
Desktop, for example, you would type the following:

\begin{Shaded}
\begin{Highlighting}[]
\KeywordTok{setwd}\NormalTok{(}\StringTok{"C:/Users/username/Desktop/Creating-maps-in-R-master/"}\NormalTok{)}
\end{Highlighting}
\end{Shaded}
If you are working in RStudio, you can create a project that will
automatically set your working directory.To do this click ``Session''
from the top toolbar and select ``Set working directory \textgreater{}
choose directory''.

It is also worth taking a look at the input data in your file browser
before opening them in R, to get a feel for them. You could try opening
the file ``london\_sport.shp'', within the ``data'' folder of the
project, in a GIS program such as QGIS (which can be freely downloaded
from the internet), for example, to get a feel for it before loading it
into R. Also note that .shp files are composed of several files for each
object: you should be able to open ``london\_sport.dbf'' in a
spreadsheet program such as LibreOffice Calc. Once you've understood
something of this input data and where it lives, it's time to open it in
R.

\subsection{Loading the spatial data}

One of the most important steps in handling spatial data with R is the
ability to read in spatial data, such as
\href{http://en.wikipedia.org/wiki/Shapefile}{shapefiles} (a common
geographical file format). There are a number of ways to do this, the
most commonly used and versatile of which is \texttt{readOGR}. This
function, from the \texttt{rgdal} package, automatically extracts
information about the projection and the attributes of data.
\texttt{rgdal} is R's interface to the ``Geospatial Abstraction Library
(GDAL)'' which is used by other open source GIS packages such as QGIS
and enables R to handle a broader range of spatial data formats. If
you've not already \emph{installed} and loaded the rgdal package (as
described above for ggplot2) do so now:

\begin{Shaded}
\begin{Highlighting}[]
\KeywordTok{library}\NormalTok{(rgdal)}
\NormalTok{sport <- }\KeywordTok{readOGR}\NormalTok{(}\DataTypeTok{dsn =} \StringTok{"data"}\NormalTok{, }\StringTok{"london_sport"}\NormalTok{)}
\end{Highlighting}
\end{Shaded}
\begin{verbatim}
## OGR data source with driver: ESRI Shapefile 
## Source: "data", layer: "london_sport"
## with 33 features and 4 fields
## Feature type: wkbPolygon with 2 dimensions
\end{verbatim}
In the code above \texttt{dsn} stands for ``data source name'' and is an
\emph{argument} of the \emph{function} \texttt{readOGR}. Note that each
new argument is separated by a comma. The \texttt{dsn} argument in this
case, specifies the directory in which the dataset is stored. R
functions have a default order of arguments, so \texttt{dsn =} does not
actually need to be typed. If there data were stored in the current
working directory, one could use \texttt{readOGR(".", "london\_sport")}.
For clarity, it is good practice to include argument names, such as
\texttt{dsn} when learning new functions.

The next argument is a \emph{character string}. This is simply the name
of the file required. There is no need to add a file extension (e.g.
\texttt{.shp}) in this case. The files beginning \texttt{london\_sport}
from the
\href{http://spatial.ly/wp-content/uploads/2013/12/spatialggplot.zip}{example
dataset} contain the borough population and the percentage of the
population participating in sporting activities and was taken from the
\href{http://data.london.gov.uk/datastore/package/active-people-survey-kpi-data-borough}{active
people survey}. The boundary data is from the
\href{http://www.ordnancesurvey.co.uk/oswebsite/opendata/}{Ordnance
Survey}.

For information about how to load different types of spatial data, the
help documentation for \texttt{readOGR} is a good place to start. This
can be accessed from within R by typing \texttt{?readOGR}. For another
worked example, in which a GPS trace is loaded, please see Cheshire and
Lovelace (2014).

\subsection{Basic plotting}

We have now created a new spatial object called ``sport'' from the
``london\_sport'' shapefile. Spatial objects are made up of a number of
different \emph{slots}, mainly the attribute \emph{slot} and the
geometry \emph{slot}. The attribute \emph{slot} can be thought of as an
attribute table and the geometry \emph{slot} is where the spatial object
(and it's attibutes) lie in space. Lets now analyse the sport object
with some basic commands:

\begin{Shaded}
\begin{Highlighting}[]
\KeywordTok{head}\NormalTok{(sport@data, }\DataTypeTok{n =} \DecValTok{2}\NormalTok{)}
\end{Highlighting}
\end{Shaded}
\begin{verbatim}
##   ons_label                 name Partic_Per Pop_2001
## 0      00AF              Bromley       21.7   295535
## 1      00BD Richmond upon Thames       26.6   172330
\end{verbatim}
\begin{Shaded}
\begin{Highlighting}[]
\KeywordTok{mean}\NormalTok{(sport$Partic_Per)}
\end{Highlighting}
\end{Shaded}
\begin{verbatim}
## [1] 20.05
\end{verbatim}
Take a look at this output and notice the table format of the data and
the column names. There are two important symbols at work in the above
block of code: the \texttt{@} symbol in the first line of code is used
to refer to the attribute \emph{slot} of the dataset; the \texttt{\$}
symbol refers to a specific variable (column name) in the attribute
\emph{slot} of the dataset, which was identified from the result of
running the first line of code. If you are using RStudio, test out the
autocompletion functionality by hitting \texttt{tab} before completing
the command - this can save you a lot of time in the long run.

The \texttt{head} function in the first line of the code above simply
means ``show the first few lines of data'', i.e.~the head. It's default
is to output the first 6 rows of the dataset (try simply
\texttt{head(sport@data)}), but we can specify the number of lines with
\texttt{n = 2} after the comma. The second line of the code above
calculates the mean value of the variable \texttt{Partic\_Per} (sports
participation per 100 people) for each of the zones in the sport object.
To explore the sport object further, try typing \texttt{nrow(sport)} and
record how many zones the dataset contains.You can also try
\texttt{ncol(sport)}.

Now we have seen something of the attribute \emph{slot} of the spatial
dataset, let us look at sport's \emph{geometry} data, which describes
where the polygons are located in space:

\begin{Shaded}
\begin{Highlighting}[]
\KeywordTok{plot}\NormalTok{(sport)  }\CommentTok{# not shown in tutorial - try it on your computer}
\end{Highlighting}
\end{Shaded}
\texttt{plot} is one of the most useful functions in R, as it changes
its behaviour depending on the input data (this is called
\emph{polymorphism} by computer scientists). Inputing another dataset
such as \texttt{plot(sport@data)} will generate an entirely different
type of plot. Thus R is intelligent at guessing what you want to do with
the data you provide it with.

R has powerful subsetting capabilities that can be accessed very
concisely using square brackets, as shown in the following example:

\begin{Shaded}
\begin{Highlighting}[]
\CommentTok{# select rows from attribute slot of sport object, where sports}
\CommentTok{# participation is less than 15.}
\NormalTok{sport@data[sport$Partic_Per < }\DecValTok{15}\NormalTok{, ]}
\end{Highlighting}
\end{Shaded}
\begin{verbatim}
##    ons_label           name Partic_Per Pop_2001
## 17      00AQ         Harrow       14.8   206822
## 21      00BB         Newham       13.1   243884
## 32      00AA City of London        9.1     7181
\end{verbatim}
The above line of code asked R to select rows from the sport object,
where sports participation is lower than 15, in this case rows 17, 21
and 32, which are Harrow, Newham and the city centre respectively. The
square brackets work as follows: anything before the comma refers to the
rows that will be selected, anything after the comma refers to the
number of columns that should be returned. For example if the dataset
had 1000 columns and you were only interested in the first two columns
you could specify \texttt{1:2} after the comma. The ``:'' symbol simply
means ``to'', i.e.~columns 1 to 2. Try experimenting with the square
brackets notation (e.g.~guess the result of
\texttt{sport@data{[}1:2, 1:3{]}} and test it): it will be useful.

So far we have been interrogating only the attribute \emph{slot}
(\texttt{@data}) of the \texttt{sport} object, but the square brackets
can also be used to subset spatial datasets, i.e.~the geometry
\emph{slot}. Using the same logic as before try to plot a subset of
zones with high sports participation.

\begin{Shaded}
\begin{Highlighting}[]
\CommentTok{# plot zones from sports object where sports participation is greater than}
\CommentTok{# 25.}
\KeywordTok{plot}\NormalTok{(sport[sport$Partic_Per > }\DecValTok{25}\NormalTok{, ])  }\CommentTok{# output not shown in tutorial}
\end{Highlighting}
\end{Shaded}
This is useful, but it would be great to see these sporty areas in
context. To do this, simply use the \texttt{add = TRUE} argument after
the initial plot. (\texttt{add = T} would also work, but we like to
spell things out in this tutorial for clarity). What does the
\texttt{col} argument refer to in the below block - it should be
obvious.

\begin{Shaded}
\begin{Highlighting}[]
\KeywordTok{plot}\NormalTok{(sport)}
\KeywordTok{plot}\NormalTok{(sport[sport$Partic_Per > }\DecValTok{25}\NormalTok{, ], }\DataTypeTok{col =} \StringTok{"blue"}\NormalTok{, }\DataTypeTok{add =} \OtherTok{TRUE}\NormalTok{)}
\end{Highlighting}
\end{Shaded}
\begin{figure}[htbp]
\centering
\includegraphics{figure/Preliminary_plot_of_London_with_areas_of_high_sports_participation_highlighted_in_blue.png}
\caption{Preliminary plot of London with areas of high
sports participation highlighted in blue}
\end{figure}

Congratulations! You have just interrogated and visualised a spatial
dataset: what kind of places have high levels of sports participation?
The map tells us. Do not worry for now about the intricacies of how this
was achieved: you have learned vital basics of how R works as a
language; we will cover this in more detail in subsequent sections.

While we are on the topic of loading data, it is worth pointing out that
R can save and load data efficiently into its own data format
(\texttt{.RData}). Try \texttt{save(sport, file = "sport.RData")} and
see what happens. If you type \texttt{rm(sport)} (which removes the
object) and then \texttt{load("sport.RData")} you should see how this
works. \texttt{sport} will disappear from the workspace and then
reappear.

\subsection{Attribute data}

All shapefiles have both attribute table and geometry data. These are
automatically loaded with \texttt{readOGR}. The loaded attribute data
can be treated in a similar way to an R
\href{http://www.statmethods.net/input/datatypes.html}{data frame}.

R delibrately hides the geometry of spatial data unless you print the
entire object (try typing \texttt{print(sport)}). Let's take a look at
the headings of sport, using the following command:
\texttt{names(sport)} Remember, the attribute data contained in spatial
objects are kept in a `slot' that can be accessed using the \texttt{@}
symbol: \texttt{sport@data}. This is useful if you do not wish to work
with the spatial components of the data at all times.

Type \texttt{summary(sport)} to get some additional information about
the sport data object. Spatial objects in R contain much additional
information:

\begin{verbatim}
summary(sport)

## Object of class SpatialPolygonsDataFrame
## Coordinates:
##       min      max
## x 503571.2 561941.1
## y 155850.8 200932.5
## Is projected: TRUE 
## proj4string :
## [+proj=tmerc +lat_0=49 +lon_0=-2 +k=0.9996012717 ....]
\end{verbatim}
The above output tells us that \texttt{sport} is a special spatial
class, in this case a \texttt{SpatialPolygonsDataFrame}, meaning it is
composed of various polygons, each of which has attributes. This is the
typical class of data found in administrative zones. The coordinates
tell us what the maximum and minimum x and y values are, for plotting.
Finally, we are told something of the coordinate reference system with
the \texttt{Is projected} and \texttt{proj4string} lines. In this case,
we have a projected system, which means it is a cartesian reference
system, relative to some point on the surface of the Earth. We will
cover reprojecting data in the next part of the tutorial.

\section{Part III: Manipulating spatial data}

It is all very well being able to load and interrogate spatial data in
R, but to compete with modern GIS packages, R must also be able to
modify these spatial objects (see
`\href{https://github.com/Pakillo/R-GIS-tutorial}{using R as a GIS}'). R
has a wide range of very powerful functions for this, many of which
reside in additional packages alluded to in the introduction.

This course is introductory so only commonly required data manipulation
tasks, \emph{reprojecting} and \emph{joining/clipping} are covered here.
We will first look at joining an aspatial dataset to our spatial object
using an attribute join. We will then cover spatial joins, whereby data
is joined to other dataset based on spatial location.

\subsection{Changing projection}

First things first, before we start data manipulation we will check the
reference system of our spatial datasets. You may have noticed the word
\texttt{proj4string} in the summary of the \texttt{sport} object above.
This represents the coordinate reference system used in the data. In
this file it has been incorrectly specified so we must change it with
the following:

\begin{Shaded}
\begin{Highlighting}[]
\KeywordTok{proj4string}\NormalTok{(sport) <- }\KeywordTok{CRS}\NormalTok{(}\StringTok{"+init=epsg:27700"}\NormalTok{)}
\end{Highlighting}
\end{Shaded}
\begin{verbatim}
## Warning: A new CRS was assigned to an object with an existing CRS:
## +proj=tmerc +lat_0=49 +lon_0=-2 +k=0.9996012717 +x_0=400000 +y_0=-100000 +ellps=airy +units=m +no_defs
## without reprojecting.
## For reprojection, use function spTransform in package rgdal
\end{verbatim}
You will see a warning. This simply states that you are changing the
coordinate reference system, not reprojecting the data. R uses epsg
codes to refer to different coordinate reference systems. Epsg:27700 is
the code for British National Grid. If we wanted to reproject the data
into something like WGS84 for latitude and longitude we would use the
following code:

\begin{Shaded}
\begin{Highlighting}[]
\NormalTok{sport.wgs84 <- }\KeywordTok{spTransform}\NormalTok{(sport, }\KeywordTok{CRS}\NormalTok{(}\StringTok{"+init=epsg:4326"}\NormalTok{))}
\end{Highlighting}
\end{Shaded}
The above line of code uses the function \texttt{spTransform}, from the
\texttt{sp} package, to convert the \texttt{sport} object into a new
form, with the Coordinate Reference System (CRS) specified as WGS84. The
different epsg codes are a bit of hassle to remember but you can search
for them at \href{http://spatialreference.org/}{spatialreference.org}.

\subsection{Attribute joins}

Attribute joins are used to link additional pieces of information to our
polygons. in the \texttt{sport} object, for example, we have 5 attribute
variables - that can be found by typing \texttt{names(sport)}. But what
happens when we want to add an additional variable from an external data
table? We will use the example of recorded crimes by borough to
demonstrate this.

To reaffirm our starting point, let's re-load the ``london\_sport''
shapefile as a new object and plot it. This is identical to the
\texttt{sport} object in the first instance, but we will give it a new
name, in case we ever need to re-use \texttt{sport}. We will call this
new object \texttt{lnd}, short for London:

\begin{Shaded}
\begin{Highlighting}[]
\KeywordTok{library}\NormalTok{(rgdal)  }\CommentTok{# ensure rgdal is loaded}
\CommentTok{# Create new object called 'lnd' from 'london_sport' shapefile}
\NormalTok{lnd <- }\KeywordTok{readOGR}\NormalTok{(}\DataTypeTok{dsn =} \StringTok{"data"}\NormalTok{, }\StringTok{"london_sport"}\NormalTok{)}
\end{Highlighting}
\end{Shaded}
\begin{verbatim}
## OGR data source with driver: ESRI Shapefile 
## Source: "data", layer: "london_sport"
## with 33 features and 4 fields
## Feature type: wkbPolygon with 2 dimensions
\end{verbatim}
\begin{Shaded}
\begin{Highlighting}[]

\KeywordTok{plot}\NormalTok{(lnd)  }\CommentTok{# plot the lnd object }
\end{Highlighting}
\end{Shaded}
\begin{figure}[htbp]
\centering
\includegraphics{figure/Plot_of_London.png}
\caption{Plot of London}
\end{figure}

\begin{Shaded}
\begin{Highlighting}[]
\KeywordTok{nrow}\NormalTok{(lnd)  }\CommentTok{# return the number of rows}
\end{Highlighting}
\end{Shaded}
\begin{verbatim}
## [1] 33
\end{verbatim}
The aspatial dataset we are going to join to the \texttt{lnd} object is
a dataset on recorded crimes, this dataset currently resides in a comma
delimited (\texttt{.csv}) file called ``mps-recordedcrime-borough'' with
each row representing a single reported crime. We are going to use a
function called \texttt{aggregate} to pre-process this dataset ready to
join to our spatial \texttt{lnd} dataset. First we will create a new
object called \texttt{crimeDat} to store this data.

\begin{Shaded}
\begin{Highlighting}[]
\CommentTok{# Create new crimeDat object from crime data It has an unusual encoding,}
\CommentTok{# hence the fileEncoding argument, usually unnecessary}
\NormalTok{crimeDat <- }\KeywordTok{read.csv}\NormalTok{(}\StringTok{"data/mps-recordedcrime-borough.csv"}\NormalTok{, }\DataTypeTok{fileEncoding =} \StringTok{"UCS-2LE"}\NormalTok{)}

\KeywordTok{head}\NormalTok{(crimeDat)  }\CommentTok{# display first 6 lines of the crimeDat object (not shown)}
\KeywordTok{summary}\NormalTok{(crimeDat$MajorText)  }\CommentTok{# summarise the column 'MajorText' for the crimeDat object}

\CommentTok{# Extract 'Theft & Handling' crimes from crimeDat object and save these as}
\CommentTok{# crimeTheft}
\NormalTok{crimeTheft <- crimeDat[}\KeywordTok{which}\NormalTok{(crimeDat$MajorText == }\StringTok{"Theft & Handling"}\NormalTok{), ]}
\KeywordTok{head}\NormalTok{(crimeTheft, }\DecValTok{2}\NormalTok{)  }\CommentTok{# take a look at the result (replace 2 with 10 to see more rows)}

\CommentTok{# Calculate the sum of the crime count for each district and save result as}
\CommentTok{# a new object}
\NormalTok{crimeAg <- }\KeywordTok{aggregate}\NormalTok{(CrimeCount ~ Spatial_DistrictName, }\DataTypeTok{FUN =} \NormalTok{sum, }\DataTypeTok{data =} \NormalTok{crimeTheft)}
\CommentTok{# Show the first two rows of the aggregated crime data}
\KeywordTok{head}\NormalTok{(crimeAg, }\DecValTok{2}\NormalTok{)}
\end{Highlighting}
\end{Shaded}
There is a lot going on in the above block of code and you should not
expect to understand all of it upon first try: simply typing the
commands and thinking briefly about the outputs is all that is needed at
this stage to improve your intuitive understanding of R. It is worth
pointing out, however, that the \texttt{\ensuremath{\sim}} symbol means
``by'': we aggregated the CrimeCount variable by the district name.

Now that we have crime data at the borough level
(\texttt{Spatial\_DistrictName}), the challenge is to join it to the
\texttt{lnd} object. We will base our join on the
\texttt{Spatial\_DistrictName} variable from the \texttt{crimeAg} object
and the \texttt{name} variable from the \texttt{lnd} object. It is not
always straight forward to join objects based on names as the names do
not always match. Let us see which names in the \texttt{crimeAg} object
match the spatial data object, \texttt{lnd}:

\begin{Shaded}
\begin{Highlighting}[]
\CommentTok{# Compare the name column in lnd to Spatial_DistrictName column in crimeAg}
\CommentTok{# to see which rows match.}
\NormalTok{lnd$name %in% crimeAg$Spatial_DistrictName}
\end{Highlighting}
\end{Shaded}
\begin{verbatim}
##  [1]  TRUE  TRUE  TRUE  TRUE  TRUE  TRUE  TRUE  TRUE  TRUE  TRUE  TRUE
## [12]  TRUE  TRUE  TRUE  TRUE  TRUE  TRUE  TRUE  TRUE  TRUE  TRUE  TRUE
## [23]  TRUE  TRUE  TRUE  TRUE  TRUE  TRUE  TRUE  TRUE  TRUE  TRUE FALSE
\end{verbatim}
\begin{Shaded}
\begin{Highlighting}[]
\CommentTok{# Return rows which do not match}
\NormalTok{lnd$name[}\KeywordTok{which}\NormalTok{(!lnd$name %in% crimeAg$Spatial_DistrictName)]}
\end{Highlighting}
\end{Shaded}
\begin{verbatim}
## [1] City of London
## 33 Levels: Barking and Dagenham Barnet Bexley Brent Bromley ... Westminster
\end{verbatim}
The first line of code above uses the \texttt{\%in\%} command to
identify which values in \texttt{lnd\$name} are also contained in the
names of the crime data. The results indicate that all but one of the
borough names matches. The second line of code tells us that it is City
of London, row 25, that is named differently in the crime data. Look at
the results (not shown here) on your computer.

\begin{Shaded}
\begin{Highlighting}[]
\CommentTok{# Discover the names of the names}
\KeywordTok{levels}\NormalTok{(crimeAg$Spatial_DistrictNam)  }\CommentTok{# not shown n tutorial}
\end{Highlighting}
\end{Shaded}
\begin{verbatim}
##  [1] "Barking and Dagenham"   "Barnet"                
##  [3] "Bexley"                 "Brent"                 
##  [5] "Bromley"                "Camden"                
##  [7] "Croydon"                "Ealing"                
##  [9] "Enfield"                "Greenwich"             
## [11] "Hackney"                "Hammersmith and Fulham"
## [13] "Haringey"               "Harrow"                
## [15] "Havering"               "Hillingdon"            
## [17] "Hounslow"               "Islington"             
## [19] "Kensington and Chelsea" "Kingston upon Thames"  
## [21] "Lambeth"                "Lewisham"              
## [23] "Merton"                 "Newham"                
## [25] "NULL"                   "Redbridge"             
## [27] "Richmond upon Thames"   "Southwark"             
## [29] "Sutton"                 "Tower Hamlets"         
## [31] "Waltham Forest"         "Wandsworth"            
## [33] "Westminster"
\end{verbatim}
\begin{Shaded}
\begin{Highlighting}[]

\CommentTok{# Rename row 25 in crimeAg to match row 25 in lnd, as suggested results form}
\CommentTok{# above}
\KeywordTok{levels}\NormalTok{(crimeAg$Spatial_DistrictName)[}\DecValTok{25}\NormalTok{] <- }\KeywordTok{as.character}\NormalTok{(lnd$name[}\KeywordTok{which}\NormalTok{(!lnd$name %in% }
    \NormalTok{crimeAg$Spatial_DistrictName)])}
\NormalTok{lnd$name %in% crimeAg$Spatial_DistrictName  }\CommentTok{# now all columns match}
\end{Highlighting}
\end{Shaded}
\begin{verbatim}
##  [1] TRUE TRUE TRUE TRUE TRUE TRUE TRUE TRUE TRUE TRUE TRUE TRUE TRUE TRUE
## [15] TRUE TRUE TRUE TRUE TRUE TRUE TRUE TRUE TRUE TRUE TRUE TRUE TRUE TRUE
## [29] TRUE TRUE TRUE TRUE TRUE
\end{verbatim}
The above code block first identified the row with the faulty name and
then renamed the level to match the \texttt{lnd} dataset. Note that we
could not rename the variable directly, as it is stored as a factor.

We are now ready to join the datasets. It is recommended to use the
\texttt{join} function in the \texttt{plyr} package but the
\texttt{merge} function could equally be used. Note that when we ask for
help for a function that is not loaded, nothing happens, indicating we
need to load it:

\begin{Shaded}
\begin{Highlighting}[]
\StringTok{`}\DataTypeTok{?}\StringTok{`}\NormalTok{(join)}
\KeywordTok{library}\NormalTok{(plyr)}
\StringTok{`}\DataTypeTok{?}\StringTok{`}\NormalTok{(join)}
\end{Highlighting}
\end{Shaded}
The documentation for join will be displayed if the plyr package is
loaded (if not, load or install and load it!). It requires all joining
variables to have the same name, so we will rename the variable to make
the join work:

\begin{Shaded}
\begin{Highlighting}[]
\KeywordTok{head}\NormalTok{(lnd$name)}
\KeywordTok{head}\NormalTok{(crimeAg$Spatial_DistrictName)  }\CommentTok{# the variables to join}
\NormalTok{crimeAg <- }\KeywordTok{rename}\NormalTok{(crimeAg, }\DataTypeTok{replace =} \KeywordTok{c}\NormalTok{(}\DataTypeTok{Spatial_DistrictName =} \StringTok{"name"}\NormalTok{))}
\KeywordTok{head}\NormalTok{(}\KeywordTok{join}\NormalTok{(lnd@data, crimeAg))  }\CommentTok{# test it works}
\end{Highlighting}
\end{Shaded}
\begin{verbatim}
## Joining by: name
\end{verbatim}
\begin{Shaded}
\begin{Highlighting}[]
\NormalTok{lnd@data <- }\KeywordTok{join}\NormalTok{(lnd@data, crimeAg)}
\end{Highlighting}
\end{Shaded}
\begin{verbatim}
## Joining by: name
\end{verbatim}
Take a look at the \texttt{lnd@data} object. You should see new
variables added, meaning the attribute join was successful.

\subsection{Clipping and spatial joins}

In addition to joining by zone name, it is also possible to do
\href{http://help.arcgis.com/en/arcgisdesktop/10.0/help/index.html\#//00080000000q000000}{spatial
joins} in R. There are three main varieties: many-to-one, where the
values of many intersecting objects contribute to a new variable in the
main table, one-to-many, or one-to-one. Because boroughs in London are
quite large, we will conduct a many-to-one spatial join. We will be
using Tube Stations as the spatial data to join, with the aim of finding
out which and how many stations are found in each London borough.

\begin{Shaded}
\begin{Highlighting}[]
\KeywordTok{library}\NormalTok{(rgdal)}
\CommentTok{# create new stations object using the 'lnd-stns' shapefile.}
\NormalTok{stations <- }\KeywordTok{readOGR}\NormalTok{(}\DataTypeTok{dsn =} \StringTok{"data"}\NormalTok{, }\DataTypeTok{layer =} \StringTok{"lnd-stns"}\NormalTok{)}
\KeywordTok{proj4string}\NormalTok{(stations)  }\CommentTok{# this is the full geographical detail.}
\KeywordTok{proj4string}\NormalTok{(lnd)}
\CommentTok{# return the bounding box of the stations object}
\KeywordTok{bbox}\NormalTok{(stations)}
\CommentTok{# return the bounding box of the lnd object}
\KeywordTok{bbox}\NormalTok{(lnd)}
\end{Highlighting}
\end{Shaded}
The above code loads the data correctly, but also shows that there are
problems with it: the Coordinate Reference System (CRS) of the stations
differs from that of our \texttt{lnd} object. OSGB 1936 (or
\href{http://spatialreference.org/ref/epsg/27700/}{EPSG 27700}) is the
official CRS for the UK, so we will convert the stations dataset to
this:

\begin{Shaded}
\begin{Highlighting}[]
\CommentTok{# create new stations27700 object which the stations object reprojected into}
\CommentTok{# OSGB36}
\NormalTok{stations27700 <- }\KeywordTok{spTransform}\NormalTok{(stations, }\DataTypeTok{CRSobj =} \KeywordTok{CRS}\NormalTok{(}\KeywordTok{proj4string}\NormalTok{(lnd)))}
\CommentTok{# overwrite the stations object with stations27700}
\NormalTok{stations <- stations27700}
\KeywordTok{rm}\NormalTok{(stations27700)  }\CommentTok{# remove the stations27700 object to clear up}
\KeywordTok{plot}\NormalTok{(lnd)}
\KeywordTok{points}\NormalTok{(stations)}
\end{Highlighting}
\end{Shaded}
\begin{figure}[htbp]
\centering
\includegraphics{figure/Sampling_and_plotting_stations.png}
\caption{Sampling and plotting stations}
\end{figure}

Now we can clearly see that the stations overlay the boroughs. The
problem is that the stations dataset is far more extensive than the
London borough dataset; so we will take a spatially determined subset of
the stations object so that they all fit within the lnd extent. This is
\emph{clipping}.

There are a number of functions that we can use to clip the stations
dataset so that only those falling within London boroughs are retained.
These include \texttt{overlay}, \texttt{sp::over}, and
\texttt{rgeos::gIntersects} (the word preceding the \texttt{::} symbol
refers to the package which the function is from). Use \texttt{?}
followed by the function to get help on each and find out which is most
appropriate. \texttt{gIntersects} can produce the same output as
\texttt{over} for basic joins (Bivand et al. 2013).

In this tutorial we will use the \texttt{over} function as it is easiest
to use. \texttt{gIntersects} can achieve the same result, but with more
lines of code. It may seem confusing that two different functions can be
used to generate the same result. However, this is a common issue in
programming; the question is finding the most appropriate solution.

\texttt{over} takes two main input arguments: the target layer (the
layer to be altered) and the source layer by which the target layer is
to be clipped. The output of \texttt{over} is a data frame of the same
dimensions as the original dataset (in this case \texttt{stations}),
except that the points which fall outside the zone of interest are set
to a value of \texttt{NA} (``no answer''). We can use this to make a
subset of the original polygons, remembering the square bracket notation
described in the first section. We create a new object, \texttt{sel}
(short for ``selection''), containing the indices of all relevant
polygons:

\begin{Shaded}
\begin{Highlighting}[]
\NormalTok{sel <- }\KeywordTok{over}\NormalTok{(stations, lnd)}
\NormalTok{stations <- stations[!}\KeywordTok{is.na}\NormalTok{(sel[, }\DecValTok{1}\NormalTok{]), ]}
\end{Highlighting}
\end{Shaded}
Typing \texttt{summary(sel)} should provide insight into how this
worked: it is a dataframe with 1801 NA values, representing zones
outside of the London polygon. Because this is a common procedure it is
actually possible to perform it with a single line of code:

\begin{Shaded}
\begin{Highlighting}[]
\NormalTok{stations <- stations[lnd, ]}
\KeywordTok{plot}\NormalTok{(stations)}
\end{Highlighting}
\end{Shaded}
\begin{figure}[htbp]
\centering
\includegraphics{figure/The_clipped_stations_dataset.png}
\caption{The clipped stations dataset}
\end{figure}

As the figure shows, only stations within the London boroughs are now
shown.

The \emph{third} way to achieve the same result uses the \texttt{rgeos}
package. This is more complex and not included in this tutorial
(interested readers can see a vignette of this, to accompany the
tutorial on
\href{http://rpubs.com/RobinLovelace/11796}{RPubs.com/Robinlovelace}).
The next section demonstrates spatial aggregation, a more advanced
version of spatial subsetting.

\subsection{Spatial aggregation}

As with R's very terse code for spatial subsetting, the base function
\texttt{aggregate} (which provides summaries of variables based on some
grouping variable) also behaves differently when the inputs are spatial
objects.

\begin{Shaded}
\begin{Highlighting}[]
\NormalTok{stations.c <- }\KeywordTok{aggregate}\NormalTok{(}\DataTypeTok{x =} \NormalTok{stations, }\DataTypeTok{by =} \NormalTok{lnd, }\DataTypeTok{FUN =} \NormalTok{length)}
\NormalTok{stations.c@data[, }\DecValTok{1}\NormalTok{]}
\end{Highlighting}
\end{Shaded}
\begin{verbatim}
##  [1] 48 22 43 18 12 13 25 24 12 46 18 20 28 32 38 19 30 25 31  7 10 38 12
## [24] 16 28 17 16 28  4  6 14 26  5
\end{verbatim}
The above code performs a number of steps in just one line:

\begin{itemize}
\item
  \texttt{aggregate} identifies which \texttt{lnd} polygon (borough)
  each \texttt{station} is located in and groups them accordingly
\item
  it counts the number of stations in each borough
\item
  a new spatial object is created and assigned the name
  \texttt{stations.c}, the count of stations
\end{itemize}
As shown below, the spatial implementation of \texttt{aggregate} can
provide summary statistics of variables. In this case we take the
variable \texttt{NUMBER} and find its mean value for the stations in
each ward.

\begin{Shaded}
\begin{Highlighting}[]
\NormalTok{stations.m <- }\KeywordTok{aggregate}\NormalTok{(stations[}\KeywordTok{c}\NormalTok{(}\StringTok{"NUMBER"}\NormalTok{)], }\DataTypeTok{by =} \NormalTok{lnd, }\DataTypeTok{FUN =} \NormalTok{mean)}
\end{Highlighting}
\end{Shaded}
For an optional advanced task, let us analyse and plot the result.

\begin{Shaded}
\begin{Highlighting}[]
\NormalTok{q <- }\KeywordTok{cut}\NormalTok{(stations.m$NUMBER, }\DataTypeTok{breaks =} \KeywordTok{c}\NormalTok{(}\KeywordTok{quantile}\NormalTok{(stations.m$NUMBER)), }\DataTypeTok{include.lowest =} \NormalTok{T)}
\KeywordTok{summary}\NormalTok{(q)}
\end{Highlighting}
\end{Shaded}
\begin{verbatim}
## [1.82e+04,1.94e+04] (1.94e+04,1.99e+04] (1.99e+04,2.05e+04] 
##                   9                   8                   8 
##  (2.05e+04,2.1e+04] 
##                   8
\end{verbatim}
\begin{Shaded}
\begin{Highlighting}[]
\NormalTok{clr <- }\KeywordTok{as.character}\NormalTok{(}\KeywordTok{factor}\NormalTok{(q, }\DataTypeTok{labels =} \KeywordTok{paste0}\NormalTok{(}\StringTok{"grey"}\NormalTok{, }\KeywordTok{seq}\NormalTok{(}\DecValTok{20}\NormalTok{, }\DecValTok{80}\NormalTok{, }\DecValTok{20}\NormalTok{))))}
\KeywordTok{plot}\NormalTok{(stations.m, }\DataTypeTok{col =} \NormalTok{clr)}
\KeywordTok{legend}\NormalTok{(}\DataTypeTok{legend =} \KeywordTok{paste0}\NormalTok{(}\StringTok{"q"}\NormalTok{, }\DecValTok{1}\NormalTok{:}\DecValTok{4}\NormalTok{), }\DataTypeTok{fill =} \KeywordTok{paste0}\NormalTok{(}\StringTok{"grey"}\NormalTok{, }\KeywordTok{seq}\NormalTok{(}\DecValTok{20}\NormalTok{, }\DecValTok{80}\NormalTok{, }\DecValTok{20}\NormalTok{)), }\StringTok{"topright"}\NormalTok{)}
\end{Highlighting}
\end{Shaded}
\begin{figure}[htbp]
\centering
\includegraphics{figure/Choropleth_map_of_mean_values_of_stations_in_each_borough.png}
\caption{Choropleth map of mean values of stations in each
borough}
\end{figure}

\begin{Shaded}
\begin{Highlighting}[]
\NormalTok{areas <- }\KeywordTok{sapply}\NormalTok{(stations.m@polygons, function(x) x@area)}
\end{Highlighting}
\end{Shaded}
This results in a simple choropleth map and a new vector containing the
area of each borough. As an additional step, try comparing the mean area
of each borough with the mean value of stations within it:
\texttt{plot(stations.m\$NUMBER, areas)}.

\subsection{Optional advanced task: aggregation with gIntersects}

As with clipping, we can also do spatial aggregation with the rgeos
package. In some ways, this method makes explicit the steps taken in
\texttt{aggregate} `under the hood'. The code is quite involved and
intimidating, so feel free to skip this stage. Working through and
thinking about it this alternative method may, however, yield dividends
if you intend to perform more sophisticated spatial analysis in R.

\begin{Shaded}
\begin{Highlighting}[]
\KeywordTok{library}\NormalTok{(rgeos)}
\end{Highlighting}
\end{Shaded}
\begin{verbatim}
## rgeos version: 0.3-2, (SVN revision 413M)
##  GEOS runtime version: 3.3.8-CAPI-1.7.8 
##  Polygon checking: TRUE
\end{verbatim}
\begin{Shaded}
\begin{Highlighting}[]
\NormalTok{int <- }\KeywordTok{gIntersects}\NormalTok{(stations, lnd, }\DataTypeTok{byid =} \OtherTok{TRUE}\NormalTok{)  }\CommentTok{# re-run the intersection query }
\KeywordTok{head}\NormalTok{(}\KeywordTok{apply}\NormalTok{(int, }\DataTypeTok{MARGIN =} \DecValTok{2}\NormalTok{, }\DataTypeTok{FUN =} \NormalTok{which))}
\NormalTok{b.indexes <- }\KeywordTok{which}\NormalTok{(int, }\DataTypeTok{arr.ind =} \OtherTok{TRUE}\NormalTok{)}
\KeywordTok{summary}\NormalTok{(b.indexes)}
\NormalTok{b.names <- lnd$name[b.indexes[, }\DecValTok{1}\NormalTok{]]}
\NormalTok{b.count <- }\KeywordTok{aggregate}\NormalTok{(b.indexes ~ b.names, }\DataTypeTok{FUN =} \NormalTok{length)}
\KeywordTok{head}\NormalTok{(b.count)}
\end{Highlighting}
\end{Shaded}
The above code first extracts the index of the row (borough) for which
the corresponding column is true and then converts this into names. The
final object created, \texttt{b.count} contains the number of station
points in each zone. According to this, Barking and Dagenham should
contain 12 station points. It is important to check the output makes
sense at every stage with R, so let's check to see this is indeed the
case with a quick plot:

\begin{Shaded}
\begin{Highlighting}[]
\KeywordTok{plot}\NormalTok{(lnd[}\KeywordTok{which}\NormalTok{(}\KeywordTok{grepl}\NormalTok{(}\StringTok{"Barking"}\NormalTok{, lnd$name)), ])}
\KeywordTok{points}\NormalTok{(stations)}
\end{Highlighting}
\end{Shaded}
\begin{figure}[htbp]
\centering
\includegraphics{figure/Train/tube_stations_in_Barking_and_Dagenham.png}
\caption{Train/tube stations in Barking and Dagenham}
\end{figure}

Now the fun part: count the points in the polygon and report back how
many there are!

We have now seen how to load, join and clip data. The second half of
this tutorial is concerned with \emph{visualisation} of the results. For
this, we will use ggplot2 and begin by looking at how it handles
non-spatial data.

\section{Part IV: Map making with ggplot2}

This third part introduces a slightly different method of creating plots
in R using the \href{http://ggplot2.org/}{ggplot2 package}, and explains
how it can make maps. The package is an implementation of the Grammar of
Graphics (Wilkinson 2005) - a general scheme for data visualisation that
breaks up graphs into semantic components such as scales and layers.
ggplot2 can serve as a replacement for the base graphics in R (the
functions you have been plotting with today) and contains a number of
default options that match good visualisation practice.

The maps we produce will not be that meaningful - the focus here is on
sound visualisation with R and not sound analysis (obviously the value
of the former diminished in the absence of the latter!) Whilst the
instructions are step by step you are encouraged to deviate from them
(trying different colours for example) to get a better understanding of
what we are doing.

\texttt{ggplot2} is one of the best documented packages in R. The full
documentation for it can be found online and it is recommended you test
out the examples on your own machines and play with them:
http://docs.ggplot2.org/current/ .

Good examples of graphs can also be found on the website
\href{http://www.cookbook-r.com/Graphs/}{cookbook-r.com}.

Load the package:

\begin{Shaded}
\begin{Highlighting}[]
\KeywordTok{library}\NormalTok{(ggplot2)}
\end{Highlighting}
\end{Shaded}
It is worth noting that the basic \texttt{plot()} function requires no
data preparation but additional effort in colour selection/adding the
map key etc. \texttt{qplot()} and \texttt{ggplot()} (from the ggplot2
package) require some additional steps to format the spatial data but
select colours and add keys etc. automatically. More on this later.

As a first attempt with ggplot2 we can create a scatter plot with the
attribute data in the `sport' object created above. Type:

\begin{Shaded}
\begin{Highlighting}[]
\NormalTok{p <- }\KeywordTok{ggplot}\NormalTok{(sport@data, }\KeywordTok{aes}\NormalTok{(Partic_Per, Pop_2001))}
\end{Highlighting}
\end{Shaded}
What you have just done is set up a ggplot object where you say where
you want the input data to come from. \texttt{sport@data} is actually a
data frame contained within the wider spatial object \texttt{sport} (the
\texttt{@} enables you to access the attribute table of the sport
shapefile). The characters inside the \texttt{aes} argument refer to the
parts of that data frame you wish to use (the variables
\texttt{Partic\_Per} and \texttt{Pop\_2001}). This has to happen within
the brackets of \texttt{aes()}, which means, roughly speaking
`aesthetics that vary'.

If you just type p and hit enter you get the error
\texttt{No layers in plot}. This is because you have not told ggplot
what you want to do with the data. We do this by adding so-called
``geoms'', in this case \texttt{geom\_point()}.

\begin{Shaded}
\begin{Highlighting}[]
\NormalTok{p + }\KeywordTok{geom_point}\NormalTok{()}
\end{Highlighting}
\end{Shaded}
\begin{figure}[htbp]
\centering
\includegraphics{figure/A_simple_ggplot.png}
\caption{A simple ggplot}
\end{figure}

Within the brackets you can alter the nature of the points. Try
something like \texttt{p + geom\_point(colour = "red", size=2)} and
experiment.

If you want to scale the points by borough population and colour them by
sports participation this is also fairly easy by adding another
\texttt{aes()} argument.

\begin{Shaded}
\begin{Highlighting}[]
\NormalTok{p + }\KeywordTok{geom_point}\NormalTok{(}\KeywordTok{aes}\NormalTok{(}\DataTypeTok{colour =} \NormalTok{Partic_Per, }\DataTypeTok{size =} \NormalTok{Pop_2001))}
\end{Highlighting}
\end{Shaded}
The real power of ggplot2 lies in its ability to add layers to a plot.
In this case we can add text to the plot.

\begin{Shaded}
\begin{Highlighting}[]
\NormalTok{p + }\KeywordTok{geom_point}\NormalTok{(}\KeywordTok{aes}\NormalTok{(}\DataTypeTok{colour =} \NormalTok{Partic_Per, }\DataTypeTok{size =} \NormalTok{Pop_2001)) + }\KeywordTok{geom_text}\NormalTok{(}\DataTypeTok{size =} \DecValTok{2}\NormalTok{, }
    \KeywordTok{aes}\NormalTok{(}\DataTypeTok{label =} \NormalTok{name))}
\end{Highlighting}
\end{Shaded}
\begin{figure}[htbp]
\centering
\includegraphics{figure/ggplot_for_text.png}
\caption{ggplot for text}
\end{figure}

This idea of layers (or geoms) is quite different from the standard plot
functions in R, but you will find that each of the functions does a lot
of clever stuff to make plotting much easier (see the documentation for
a full list).

The following steps will create a map to show the percentage of the
population in each London Borough who regularly participate in sports
activities.

\subsection{``Fortifying'' spatial objects for ggplot2 maps}

To get the shapefiles into a format that can be plotted we have to use
the \texttt{fortify()} function. Spatial objects in R have a number of
slots containing the various items of data (polygon geometry,
projection, attribute information) associated with a shapefile. Slots
can be thought of as shelves within the data object that contain the
different attributes. The ``polygons'' slot contains the geometry of the
polygons in the form of the XY coordinates used to draw the polygon
outline. The generic plot function can work out what to do with these,
ggplot2 cannot. We therefore need to extract them as a data frame. The
fortify function was written specifically for this purpose. For this to
work, either \texttt{maptools} or \texttt{rgeos} packages must be
installed.

\begin{Shaded}
\begin{Highlighting}[]
\NormalTok{sport.f <- }\KeywordTok{fortify}\NormalTok{(sport, }\DataTypeTok{region =} \StringTok{"ons_label"}\NormalTok{)}
\end{Highlighting}
\end{Shaded}
This step has lost the attribute information associated with the sport
object. We can add it back using the merge function (this performs a
data join). To find out how this function works look at the output of
typing \texttt{?merge}.

\begin{Shaded}
\begin{Highlighting}[]
\NormalTok{sport.f <- }\KeywordTok{merge}\NormalTok{(sport.f, sport@data, }\DataTypeTok{by.x =} \StringTok{"id"}\NormalTok{, }\DataTypeTok{by.y =} \StringTok{"ons_label"}\NormalTok{)}
\end{Highlighting}
\end{Shaded}
Take a look at the \texttt{sport.f} object to see its contents. You
should see a large data frame containing the latitude and longitude
(they are actually Easting and Northing as the data are in British
National Grid format) coordinates alongside the attribute information
associated with each London Borough. If you type \texttt{print(sport.f)}
you will see just how many coordinate pairs are required! To keep the
output to a minimum, take a peek at the object using just the
\texttt{head} command:

\begin{Shaded}
\begin{Highlighting}[]
\KeywordTok{head}\NormalTok{(sport.f[, }\DecValTok{1}\NormalTok{:}\DecValTok{8}\NormalTok{])}
\end{Highlighting}
\end{Shaded}
\begin{verbatim}
##     id   long    lat order  hole piece  group           name
## 1 00AA 531027 181611     1 FALSE     1 00AA.1 City of London
## 2 00AA 531555 181659     2 FALSE     1 00AA.1 City of London
## 3 00AA 532136 182198     3 FALSE     1 00AA.1 City of London
## 4 00AA 532946 181895     4 FALSE     1 00AA.1 City of London
## 5 00AA 533411 182038     5 FALSE     1 00AA.1 City of London
## 6 00AA 533843 180794     6 FALSE     1 00AA.1 City of London
\end{verbatim}
It is now straightforward to produce a map using all the built in tools
(such as setting the breaks in the data) that ggplot2 has to offer.
\texttt{coord\_equal()} is the equivalent of asp=T in regular plots with
R:

\begin{Shaded}
\begin{Highlighting}[]
\NormalTok{Map <- }\KeywordTok{ggplot}\NormalTok{(sport.f, }\KeywordTok{aes}\NormalTok{(long, lat, }\DataTypeTok{group =} \NormalTok{group, }\DataTypeTok{fill =} \NormalTok{Partic_Per)) + }\KeywordTok{geom_polygon}\NormalTok{() + }
    \KeywordTok{coord_equal}\NormalTok{() + }\KeywordTok{labs}\NormalTok{(}\DataTypeTok{x =} \StringTok{"Easting (m)"}\NormalTok{, }\DataTypeTok{y =} \StringTok{"Northing (m)"}\NormalTok{, }\DataTypeTok{fill =} \StringTok{"% Sport Partic."}\NormalTok{) + }
    \KeywordTok{ggtitle}\NormalTok{(}\StringTok{"London Sports Participation"}\NormalTok{)}
\end{Highlighting}
\end{Shaded}
Now, just typing \texttt{Map} should result in your first ggplot-made
map of London! There is a lot going on in the code above, so think about
it line by line: what have each of the elements of code above been
designed to do? Also note how the \texttt{aes()} components can be
combined into one set of brackets after \texttt{ggplot}, that has
relevance for all layers, rather than being broken into separate parts
as we did above. The different plot functions still know what to do with
these. The \texttt{group=group} points ggplot to the group column added
by \texttt{fortify()} and it identifies the groups of coordinates that
pertain to individual polygons (in this case London Boroughs).

The default colours are really nice but we may wish to produce the map
in black and white, which should produce a map like that shown below
(and try changing the colors):

\begin{Shaded}
\begin{Highlighting}[]
\NormalTok{Map + }\KeywordTok{scale_fill_gradient}\NormalTok{(}\DataTypeTok{low =} \StringTok{"white"}\NormalTok{, }\DataTypeTok{high =} \StringTok{"black"}\NormalTok{)}
\end{Highlighting}
\end{Shaded}
\begin{figure}[htbp]
\centering
\includegraphics{figure/Greyscale_map.png}
\caption{Greyscale map}
\end{figure}

Saving plot images is also easy. You just need to use \texttt{ggsave}
after each plot, e.g. \texttt{ggsave("my\_map.pdf")} will save the map
as a pdf, with default settings. For a larger map, you could try the
following:

\begin{Shaded}
\begin{Highlighting}[]
\KeywordTok{ggsave}\NormalTok{(}\StringTok{"my_large_plot.png"}\NormalTok{, }\DataTypeTok{scale =} \DecValTok{3}\NormalTok{, }\DataTypeTok{dpi =} \DecValTok{400}\NormalTok{)}
\end{Highlighting}
\end{Shaded}
\subsection{Adding base maps to ggplot2 with ggmap}

\href{http://journal.r-project.org/archive/2013-1/kahle-wickham.pdf}{ggmap}
is a package that uses the ggplot2 syntax as a template to create maps
with image tiles taken from map servers such as Google and
\href{http://www.openstreetmap.org/}{OpenStreetMap}:

\begin{Shaded}
\begin{Highlighting}[]
\KeywordTok{library}\NormalTok{(ggmap)  }\CommentTok{# you may have to use install.packages to install it first}
\end{Highlighting}
\end{Shaded}
The \texttt{sport} object loaded previously is in British National Grid
but the ggmap image tiles are in WGS84. We therefore need to use the
sport.wgs84 object created in the reprojection operation earlier.

The first job is to calculate the bounding box (bb for short) of the
sport.wgs84 object to identify the geographic extent of the image tiles
that we need.

\begin{Shaded}
\begin{Highlighting}[]
\NormalTok{b <- }\KeywordTok{bbox}\NormalTok{(sport.wgs84)}
\NormalTok{b[}\DecValTok{1}\NormalTok{, ] <- (b[}\DecValTok{1}\NormalTok{, ] - }\KeywordTok{mean}\NormalTok{(b[}\DecValTok{1}\NormalTok{, ])) * }\FloatTok{1.05} \NormalTok{+ }\KeywordTok{mean}\NormalTok{(b[}\DecValTok{1}\NormalTok{, ])}
\NormalTok{b[}\DecValTok{2}\NormalTok{, ] <- (b[}\DecValTok{2}\NormalTok{, ] - }\KeywordTok{mean}\NormalTok{(b[}\DecValTok{2}\NormalTok{, ])) * }\FloatTok{1.05} \NormalTok{+ }\KeywordTok{mean}\NormalTok{(b[}\DecValTok{2}\NormalTok{, ])}
\CommentTok{# scale longitude and latitude (increase bb by 5% for plot) replace 1.05}
\CommentTok{# with 1.xx for an xx% increase in the plot size}
\end{Highlighting}
\end{Shaded}
This is then fed into the \texttt{get\_map} function as the location
parameter. The syntax below contains 2 functions. \texttt{ggmap} is
required to produce the plot and provides the base map data.

\begin{Shaded}
\begin{Highlighting}[]
\NormalTok{lnd.b1 <- }\KeywordTok{ggmap}\NormalTok{(}\KeywordTok{get_map}\NormalTok{(}\DataTypeTok{location =} \NormalTok{b))}
\end{Highlighting}
\end{Shaded}
\begin{verbatim}
## Warning: bounding box given to google - spatial extent only approximate.
\end{verbatim}
In much the same way as we did above we can then layer the plot with
different geoms.

First fortify the sport.wgs84 object and then merge with the required
attribute data (we already did this step to create the sport.f object).

\begin{Shaded}
\begin{Highlighting}[]
\NormalTok{sport.wgs84.f <- }\KeywordTok{fortify}\NormalTok{(sport.wgs84, }\DataTypeTok{region =} \StringTok{"ons_label"}\NormalTok{)}
\NormalTok{sport.wgs84.f <- }\KeywordTok{merge}\NormalTok{(sport.wgs84.f, sport.wgs84@data, }\DataTypeTok{by.x =} \StringTok{"id"}\NormalTok{, }\DataTypeTok{by.y =} \StringTok{"ons_label"}\NormalTok{)}
\end{Highlighting}
\end{Shaded}
We can now overlay this on our base map.

\begin{Shaded}
\begin{Highlighting}[]
\NormalTok{lnd.b1 + }\KeywordTok{geom_polygon}\NormalTok{(}\DataTypeTok{data =} \NormalTok{sport.wgs84.f, }\KeywordTok{aes}\NormalTok{(}\DataTypeTok{x =} \NormalTok{long, }\DataTypeTok{y =} \NormalTok{lat, }\DataTypeTok{group =} \NormalTok{group, }
    \DataTypeTok{fill =} \NormalTok{Partic_Per), }\DataTypeTok{alpha =} \FloatTok{0.5}\NormalTok{)}
\end{Highlighting}
\end{Shaded}
The code above contains a lot of parameters. Use the ggplot2 help pages
to find out what they are. The resulting map looks okay, but it would be
improved with a simpler base map in black and white. A design firm
called stamen provide the tiles we need and they can be brought into the
plot with the \texttt{get\_map} function:

\begin{Shaded}
\begin{Highlighting}[]
\NormalTok{lnd.b2 <- }\KeywordTok{ggmap}\NormalTok{(}\KeywordTok{get_map}\NormalTok{(}\DataTypeTok{location =} \NormalTok{b, }\DataTypeTok{source =} \StringTok{"stamen"}\NormalTok{, }\DataTypeTok{maptype =} \StringTok{"toner"}\NormalTok{, }
    \DataTypeTok{crop =} \OtherTok{TRUE}\NormalTok{))}
\end{Highlighting}
\end{Shaded}
We can then produce the plot as before.

\begin{Shaded}
\begin{Highlighting}[]
\NormalTok{lnd.b2 + }\KeywordTok{geom_polygon}\NormalTok{(}\DataTypeTok{data =} \NormalTok{sport.wgs84.f, }\KeywordTok{aes}\NormalTok{(}\DataTypeTok{x =} \NormalTok{long, }\DataTypeTok{y =} \NormalTok{lat, }\DataTypeTok{group =} \NormalTok{group, }
    \DataTypeTok{fill =} \NormalTok{Partic_Per), }\DataTypeTok{alpha =} \FloatTok{0.5}\NormalTok{)}
\end{Highlighting}
\end{Shaded}
Finally, if we want to increase the detail of the base map, get\_map has
a zoom parameter.

\begin{Shaded}
\begin{Highlighting}[]
\NormalTok{lnd.b3 <- }\KeywordTok{ggmap}\NormalTok{(}\KeywordTok{get_map}\NormalTok{(}\DataTypeTok{location =} \NormalTok{b, }\DataTypeTok{source =} \StringTok{"stamen"}\NormalTok{, }\DataTypeTok{maptype =} \StringTok{"toner"}\NormalTok{, }
    \DataTypeTok{crop =} \OtherTok{TRUE}\NormalTok{, }\DataTypeTok{zoom =} \DecValTok{11}\NormalTok{))}

\NormalTok{lnd.b3 + }\KeywordTok{geom_polygon}\NormalTok{(}\DataTypeTok{data =} \NormalTok{sport.wgs84.f, }\KeywordTok{aes}\NormalTok{(}\DataTypeTok{x =} \NormalTok{long, }\DataTypeTok{y =} \NormalTok{lat, }\DataTypeTok{group =} \NormalTok{group, }
    \DataTypeTok{fill =} \NormalTok{Partic_Per), }\DataTypeTok{alpha =} \FloatTok{0.5}\NormalTok{)}
\end{Highlighting}
\end{Shaded}
\begin{figure}[htbp]
\centering
\includegraphics{figure/Basemap_3.png}
\caption{Basemap 3}
\end{figure}

\subsection{Advanced Task: Faceting for Maps}

\begin{Shaded}
\begin{Highlighting}[]
\KeywordTok{library}\NormalTok{(reshape2)  }\CommentTok{# this may not be installed. }
\CommentTok{# If not install it, or skip the next two steps}
\end{Highlighting}
\end{Shaded}
Load the data - this shows historic population values between 1801 and
2001 for London, again from the London data store.

\begin{Shaded}
\begin{Highlighting}[]
\NormalTok{london.data <- }\KeywordTok{read.csv}\NormalTok{(}\StringTok{"data/census-historic-population-borough.csv"}\NormalTok{)}
\end{Highlighting}
\end{Shaded}
``Melt'' the data so that the columns become rows.

\begin{Shaded}
\begin{Highlighting}[]
\NormalTok{london.data.melt <- }\KeywordTok{melt}\NormalTok{(london.data, }\DataTypeTok{id =} \KeywordTok{c}\NormalTok{(}\StringTok{"Area.Code"}\NormalTok{, }\StringTok{"Area.Name"}\NormalTok{))}
\end{Highlighting}
\end{Shaded}
Only do this step if reshape and melt failed

\begin{Shaded}
\begin{Highlighting}[]
\NormalTok{london.data.melt <- }\KeywordTok{read.csv}\NormalTok{(}\StringTok{"london_data_melt.csv"}\NormalTok{)}
\end{Highlighting}
\end{Shaded}
Merge the population data with the London borough geometry contained
within our sport.f object.

\begin{Shaded}
\begin{Highlighting}[]
\NormalTok{plot.data <- }\KeywordTok{merge}\NormalTok{(sport.f, london.data.melt, }\DataTypeTok{by.x =} \StringTok{"id"}\NormalTok{, }\DataTypeTok{by.y =} \StringTok{"Area.Code"}\NormalTok{)}
\end{Highlighting}
\end{Shaded}
Reorder this data (ordering is important for plots).

\begin{Shaded}
\begin{Highlighting}[]
\NormalTok{plot.data <- plot.data[}\KeywordTok{order}\NormalTok{(plot.data$order), ]}
\end{Highlighting}
\end{Shaded}
We can now use faceting to produce one map per year (this may take a
little while to appear).

\begin{Shaded}
\begin{Highlighting}[]
\KeywordTok{ggplot}\NormalTok{(}\DataTypeTok{data =} \NormalTok{plot.data, }\KeywordTok{aes}\NormalTok{(}\DataTypeTok{x =} \NormalTok{long, }\DataTypeTok{y =} \NormalTok{lat, }\DataTypeTok{fill =} \NormalTok{value, }\DataTypeTok{group =} \NormalTok{group)) + }
    \KeywordTok{geom_polygon}\NormalTok{() + }\KeywordTok{geom_path}\NormalTok{(}\DataTypeTok{colour =} \StringTok{"grey"}\NormalTok{, }\DataTypeTok{lwd =} \FloatTok{0.1}\NormalTok{) + }\KeywordTok{coord_equal}\NormalTok{() + }
    \KeywordTok{facet_wrap}\NormalTok{(~variable)}
\end{Highlighting}
\end{Shaded}
\begin{figure}[htbp]
\centering
\includegraphics{figure/Faceted_map.png}
\caption{Faceted map}
\end{figure}

Again there is a lot going on here so explore the documentation to make
sure you understand it. Try out different colour values as well.

Add a title and replace the axes names with ``easting'' and ``northing''
and save your map as a pdf.

\section{Part V: Taking spatial data analysis in R further}

The skills you have learned in this tutorial are applicable to a very
wide range of datasets, spatial or not. Often experimentation is the
most rewarding learning method, rather than just searching for the
`best' way of doing something (Kabakoff, 2011). We recommend you play
around with your own data.

If you would like to learn more about R's spatial functionalities,
including more exercises on loading, saving and manipulating data, we
recommend a slightly longer and more advanced tutorial (Cheshire and
Lovelace, 2014). An up-to-date repository of this project, including
example dataset and all the code used to compile the tutorial, can be
found on its GitHub page:
\href{https://github.com/geocomPP/sdvwR}{github.com/geocomPP/sdvwR}.
Another advanced tutorial is ``Using spatial data'', which has example
code and data that can be downloaded from the
\href{http://www.edii.uclm.es/~useR-2013//Tutorials/Bivand.html}{useR
2013 conference page}. Such lengthy tutorials are worth doing to think
about spatial data in R systematically, rather than seeing R as a
discrete collection of functions. In R the whole is greater than the sum
of its parts.

The supportive online communities surrounding large open source programs
such as R are one of their greatest assets, so we recommend you become
an active
``\href{http://blog.cleverelephant.ca/2013/10/being-open-source-citizen.html}{open
source citizen}'' rather than a passive consumer (Ramsey \& Dubovsky,
2013).

This does not necessarily mean writing R source code - it can simply
mean helping others use R. We therefore conclude the tutorial with a
list of resources that will help you further sharpen you R skills, find
help and contribute to the growing online R community:

\begin{itemize}
\item
  R's homepage hosts a wealth of
  \href{http://cran.r-project.org/manuals.html}{official} and
  \href{http://cran.r-project.org/other-docs.html}{contributed} guides.
\item
  Stack Exchange and GIS Stack Exchange groups - try searching for
  ``{[}R{]}''. If your issue has not been not been addressed yet, you
  could post a polite question.
\item
  R's \href{http://www.r-project.org/mail.html}{mailing lists} - the
  R-sig-geo list may be of particular interest here.
\end{itemize}
Books: despite the strength of R's online community, nothing beats a
physical book for concentrated learning. We would particularly recommend
the following:

\begin{itemize}
\item
  ggplot2: elegant graphics for data analysis (Wickham 2009)
\item
  Bivand et al. (2013) Provide a dense and detailed overview of spatial
  data analysis in an updated version of the book by the developers of
  many of R's spatial functions.
\item
  Kabacoff (2011) is a more general R book; it has many fun worked
  examples.
\end{itemize}
\section{R quick reference}

\texttt{\#}: comments all text until line end

\texttt{x \textless{}- 3}: create new object, called x, and assign value
of 3

\texttt{help(plot)}: ask R for basic help on function. Replace
\texttt{plot} with any function.

\texttt{?plot}: same as above

\texttt{library(ggplot2)}: load a package (replace \texttt{ggplot2} with
your package name)

\texttt{install.packages("ggplot2")}: install package - note quotation
marks

\texttt{setwd("C:/Users/username/Desktop/")}: set R's \emph{working
directory} (set it to your project's folder)

\texttt{nrow(df)}: count the number of rows in the object \texttt{df}

\texttt{summary(df)}: summary statistics of the object \texttt{df}

\texttt{head(df)}: display first 6 lines of object \texttt{df}

\texttt{plot(df)}: plot object \texttt{df}

\texttt{save(df, "C:/Users/username/Desktop/" )}: save df object to
specified location

\texttt{rm(df)}: remove the \texttt{df} object

\texttt{proj4string(df)}: set coordinate reference system of \texttt{df}
object

\texttt{spTransform(df, CRS("+init=epsg:4326")}: reproject \texttt{df}
object to WGS84

\section{Aknowledgements}

Many thanks to Rachel Oldroyed and Alistair Leak who helped demonstrate
these materials to participants on the NCRM short corses for which this
tutorial was developed. Amy O'Neill helped organise the course and
helped channel feedback from participants. The final thanks is to the
participants themselves who trialed earlier versions of this tutorial
and provided very useful feedback.

\newpage \section{References}

Bivand, R. S., Pebesma, E. J., \& Rubio, V. G. (2008). Applied spatial
data: analysis with R. Springer.

Cheshire, J. \& Lovelace, R. (2014). Manipulating and visualizing
spatial data with R. Book chapter in Press.

Harris, R. (2012). A Short Introduction to R.
\href{http://www.social-statistics.org/}{social-statistics.org}.

Johnson, P. E. (2013). R Style. An Rchaeological Commentary. The
Comprehensive R Archive Network.

Kabacoff, R. (2011). R in Action. Manning Publications Co.

Ramsey, P., \& Dubovsky, D. (2013). Geospatial Software's Open Future.
GeoInformatics, 16(4).

Torfs and Brauer (2012). A (very) short Introduction to R. The
Comprehensive R Archive Network.

Wickham, H. (2009). ggplot2: elegant graphics for data analysis.
Springer.

Wilkinson, L. (2005). The grammar of graphics. Springer.

\begin{Shaded}
\begin{Highlighting}[]
\KeywordTok{source}\NormalTok{(}\StringTok{"latex/rmd2pdf.R"}\NormalTok{)  }\CommentTok{# convert .Rmd to .tex file}
\end{Highlighting}
\end{Shaded}

\end{document}
